\section{Summary}

\subsection{Conclusions Current Model}
The Neurovascular coupling (NVC) matlab code version 0.1 is able to describe the relationship between a neural input signal into the \gls{SC} and the dilation or constriction of the bloodvessel. \\

An input signal (neuronal \gls{K} pulse) is taken up by the astrocyte, which repolarize its cell membrane. This repolarization results in a efflux of \gls{K} of the astrocyte in the PVS. This increase in \gls{K} in de PVS causes the opening of the KIR channel in the SMC and to extrude more \gls{K} in the PVS. This hyperpolarization closes the voltage operated calcium channels (VOCC), preventing the influx of \gls{Ca}. The decreased cytosolic \gls{Ca} in the SMC then causes dilation of the blood vessel.\\




\subsection{Further Work} \label{sec:furtherwork}

There are several pathways which have to be included in order to complete and extend the existing model. These pathways are listed below:
\begin{itemize}
\item A MATLAB model of the AA and 20-HETE pathway which has been developed needs to be included in the model. This model also includes the \gls{Ca} dynamics in the AC and therefore is an important contribution to the existing model
\item The NO (Nitric oxide) model needs to be included in this model
\item A general pH model needs to be included for the SMC and later on also the Astrocyte and EC.
\item And also the AA \& 20-HETE model needs to be included
\end{itemize}


\subsection*{Acknowledgement}
May thanks to Christine French, who helped transfer our chaotic sentences into readable English! 

%At the moment the epoxyeicosatrienoic acid (EET) and the arachidonic acid (AA) pathway are not included in the astrocyte model.\\