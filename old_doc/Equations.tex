\section{Equations}\label{sec:equations}
\todo[inline]{Some units need to be corrected in this documentation!}

\subsection{The Neuron and Astrocyte Model}

\subsubsection*{Input signals} \label{sec:InputSignal}

Neuronal \gls{K} input signal (dim.less):
\begin{equation}
f_{\text{K/Na}}(t) = 
	\begin{cases} 
		F_{\text{input}} \dfrac{(\alpha_n + \beta_n-1)!}{(\alpha_n-1)!(\beta_n-1)!} \left( \dfrac{1-(t-t_{0})}{\Delta t_2}\right) ^{\beta_n -1} \left( \dfrac{t-t_{0}}{\Delta t_2}\right) ^{\alpha_n -1}, & \text{for}\: t_{0} \leq t < t_{1} \\
		-F_{\text{input}}, & \text{for}\: t_{2} \leq t \leq t_{3} \\	
		0, & \text{otherwise}\\	
	\end{cases} 
\end{equation}

End of neuronal pulse (s):
\begin{equation}
	t_{1} = t_{0} + \Delta t
\end{equation}

Start of back-buffering (s):
\begin{equation}
	t_{2} = t_{0} + \Delta t_1
\end{equation}

End of back buffering (s):
\begin{equation}
	t_{3} = t_{1} + \Delta t_1
\end{equation}


\begin{table}[h!]
\centering
\begin{tabular}{ p{0.09\linewidth}  >{\footnotesize} p{0.5\linewidth}  >{\footnotesize} p{0.27\linewidth} >{\footnotesize} p{0.03\linewidth} }
\hline
$F_{\text{input}}$ 		& amplitude scaling factor 		& 2.5 		& ME\footnotemark[1]  \\ % on the basis of Filosa2008
$\alpha_n$ 				& beta distribution constant	& 2 		& ME  \\ % on the basis of Ostby2009
$\beta_n$ 				& beta distribution constant	& 5 		& ME  \\ % on the basis of Ostby2009
$t_{0}$ 				& start of neuronal activation	& 400 s 	& ME  \\
$\Delta t_1$ 			& length of neuronal activation & 200 s 	& ME  \\
$\Delta t_2$ 			& time-scaling factor			& 10 s		& \citep{Ostby2009}   \\
\hline
\end{tabular}
\end{table}
\footnote[1]{Model Estimation}

\subsubsection*{Scaling}
\gls{AC} volume-area ratio (in m):
\begin{equation} \label{eq:R_k}
	\begin{aligned}
	\dfrac{\mathrm{d}R_k}{\mathrm{d}t}= L_p([\Na]_k + [\K]_k + [\Cl]_k + [\HCO]_k - [\Na]_s - [\K]_s - [\Cl]_s - [\HCO]_s + \frac{X_k}{R_k})
	\end{aligned}
\end{equation}
%
\gls{SC} volume-surface ratio  (in m):
\begin{equation} \label{eq:R_tot}
R_s = R_{\text{tot}} - R_k  
\end{equation}

%Total volume-surface ratio (m):
%\begin{equation}
%	R_{\text{tot}} = \frac{V_{sc} + V_{k}}{A_k}
%\end{equation}

\begin{table}[h!]
\centering
\begin{tabular}{ p{0.09\linewidth}  >{\footnotesize} p{0.5\linewidth}  >{\footnotesize} p{0.27\linewidth} >{\footnotesize} p{0.03\linewidth} }
\hline
$L_p$ 			& total water permeability per unit area of the astrocyte  & 2.1\e{-9} \mperuMs &  \cite{Ostby2009}\footnotemark[2]  \\
$X_k$			& Number of negatively charged impermeable ions trapped within the astrocyte divided by the astrocyte membrane area								& 12.41$\times$10$^{-3}$ \uMm & \cite{Ostby2009}  \\
$R_{\text{tot}}$ 		& Total volume surface ratio AC + SC $ ((V_{sc} + V_{k})/A_k)  $ 		& 8.79$\times$10$^{-8}$ \m & \cite{Ostby2009}\footnotemark[2]  \\
%$V_{sc}$ 		& \gls{SC} volume   & 
$A_k$			& characteristic exchange surface area 	& 3.7\e{-9} m$^2$ & \citep{Ostby2009}\footnotemark[3]\\
\hline
\end{tabular}
\end{table}
\footnotetext[2]{corrected value/unit obtained from \texttt{CellML}}
\footnotetext[3]{corrected value/unit obtained from communication with author}
\subsubsection*{Conservation Equations}
\paragraph{Synaptic Cleft}~\\
%
\gls{K} concentration in the \gls{SC} (times the \gls{SC} volume-area ratio $R_s$; in \uMm):
\begin{equation} \label{eq:KEx}
\dfrac{\mathrm{d}N_{\text{K},s}}{\mathrm{d}t}= k_C f(t) -\dfrac{\mathrm{d}N_{\text{K},k}}{\mathrm{d}t} - J_{\text{BK},k}\,; \text{~~~} [\K]_s = \frac{N_{\text{K},s}}{R_s}
\end{equation}
%
\gls{Na} concentration in the \gls{SC}  (times the \gls{SC} volume-area ratio $R_s$; in \uMm):
\begin{equation} \label{eq:NaEx}
\dfrac{\mathrm{d}N_{\text{Na},s}}{\mathrm{d}t}= - k_C f(t) -\dfrac{\mathrm{d}N_{\text{Na},k}}{\mathrm{d}t}\,; \text{~~~} [\Na]_s = \frac{N_{\text{Na},s}}{R_s}
\end{equation}
%
\gls{HCO3} concentration in the SC  (times the \gls{SC} volume-area ratio $R_s$; in \uMm):
\begin{equation} \label{eq:HCOEx}
\dfrac{\mathrm{d}N_{\text{HCO}_3,s}}{\mathrm{d}t}=-\dfrac{\mathrm{d}N_{\text{HCO}_3,k}}{\mathrm{d}t}\,; \text{~~~} [\HCO]_s = \frac{N_{\text{HCO}_3,s}}{R_s}
\end{equation}
\begin{table}[h!]
\centering
\begin{tabular}{ p{0.09\linewidth}  >{\footnotesize} p{0.5\linewidth}  >{\footnotesize} p{0.27\linewidth} >{\footnotesize} p{0.03\linewidth} }
\hline
$ k_C $  & Input scaling parameter & 7.35$\times$10$^{-5}$ \muMps & \cite{Ostby2009} \\
\hline
\end{tabular}
\end{table}

\paragraph{Astrocyte}~\\
%
\gls{K} concentration in the AC  (times the AC volume-area ratio $R_k$; in \uMm):
\begin{equation} \label{eq:KInt}
\dfrac{\mathrm{d}N_{\text{K},k}}{\mathrm{d}t}=- J_{\text{K},k} + 2 J_{\text{NaK},k} + J_{NKCC1_{k}} +  J_{KCC1_{k}} - J_{\text{BK},k} \,; \text{~~~} [\K]_k = \frac{N_{\text{K},k}}{R_k}
\end{equation}
%
\gls{Na} concentration in the AC  (times the AC volume-area ratio $R_k$; in \uMm):
\begin{equation} \label{eq:NaInt}
\dfrac{\mathrm{d}N_{\text{Na},k}}{\mathrm{d}t}=-J_{\text{Na},k} - 3 J_{\text{NaK},k} + J_{NKCC1_{k}} +  J_{\text{NBC},k}\,; \text{~~~} [\Na]_k = \frac{N_{\text{Na},k}}{R_k}
\end{equation}
%
\gls{HCO3} concentration in the AC  (times the AC volume-area ratio $R_k$; in \uMm):
\begin{equation} \label{eq:HCOInt}
\dfrac{\mathrm{d}N_{\text{HCO}_3,k}}{\mathrm{d}t}= 2 J_{\text{NBC},k}\,; \text{~~~} [\HCO]_k = \frac{N_{\text{HCO}_3,k}}{R_k}
\end{equation}
%
\gls{Cl} concentration in the AC  (times the AC volume-area ratio $R_k$; in \uMm):
\begin{equation} \label{eq:ClInt}
\dfrac{\mathrm{d}N_{\text{Cl},k}}{\mathrm{d}t}= \dfrac{\mathrm{d}N_{\text{Na},k}}{\mathrm{d}t} + \dfrac{\mathrm{d}N_{\text{K},k}}{\mathrm{d}t} - \dfrac{\mathrm{d}N_{\text{HCO}_3,k}}{\mathrm{d}t}\,; \text{~~~} [\Cl]_k = \frac{N_{\text{Cl},k}}{R_k}
\end{equation}
%
Open probability of the BK channel (non-dim.):
\begin{equation} \label{eq:dwkdt}
\frac{\mathrm{d}w_{k}}{\mathrm{d}t} = \phi_{w} \left(w_{\infty}-w_{k} \right) 
\end{equation}
%
\paragraph{Perivascular Space}~\\
\gls{K} concentration in the PVS  (in \uM):
\begin{equation} \label{eq:K_p}
\dfrac{\mathrm{d}K_{p}}{\mathrm{d}t}= \frac{J_{\text{BK},k}}{R_k R_{pk}} + \frac{J_{\text{KIR},i}}{R_{ps}} - R_{\text{decay}}([\K]_p - [\K]_{p,\min});
\end{equation}
%
\begin{table}[h!]
\centering
\begin{tabular}{ p{0.09\linewidth}  >{\footnotesize} p{0.5\linewidth}  >{\footnotesize} p{0.27\linewidth} >{\footnotesize} p{0.03\linewidth} }
\hline
$ R_{pk} $  & Volume ratio of PVS to AC & 10$^{-3}$ [-] & \cite{Nagelhus1999} \\
$ R_{ps} $  & Volume ratio of PVS to SMC & 10$^{-3}$ [-] & \cite{Nagelhus1999} \\
$ R_{\text{decay}} $  & Decay rate & 0.05 s$^{-1}$ & M.E. \\
$ [\K]_{p,\min} $  & min \gls{K} concentration & 3 $\times$ 10$^{3}$ \textmu M & M.E. \\
\hline
\end{tabular}
\end{table}
\subsubsection*{Fluxes}\label{sec:EqNeAcflux}~\\ 
%
\gls{K} flux (times the AC volume-area ratio $R_k$; in \uMmps): 
\begin{equation} \label{eq:J_K}
J_{\text{K},k}=\frac{g_{K_{k}}}{F}(v_k - E_{\text{K},k})
\end{equation}
%
\gls{Na} flux (times the AC volume-area ratio $R_k$; in \uMmps):
\begin{equation} \label{eq:J_Na}
J_{\text{Na},k}=\frac{g_{Na_{k}}}{F}(v_k - E_{\text{Na},k})
\end{equation}
%
\gls{Na} and \gls{HCO3} flux through the NBC channel  (times the AC volume-area ratio $R_k$; in \uMmps): 
\begin{equation} \label{eq:J_NBC}
J_{\text{NBC},k}=\frac{g_{\text{NBC},k}}{F}\left(  v_k -E_{\text{NBC},k}  \right)
\end{equation}
%
\gls{Cl} and \gls{K} flux through the KCC1 channel  (times the AC volume-area ratio $R_k$; in \uMmps): 
\begin{equation} \label{eq:J_KCC1}
J_{\text{KCC1},k}=C_{input}\frac{g_{\text{KCC1},k}}{F}\frac{R_{\text{gas}}T}{F}ln \left(\frac{[\K]_s [\Cl]_s }{K_k [\Cl]_k}\right)
\end{equation}
%
\gls{Na}, \gls{K} and \gls{Cl} flux through the NKCC1 channel   (times the AC volume-area ratio $R_k$; in \uMmps): 
\begin{equation} \label{eq:J_NKCC1}
J_{\text{NKCC1},k}=C_{input}\frac{g_{\text{NKCC1},k}}{F}\frac{R_{\text{gas}}T}{F}ln \left(\frac{Na_s [\K]_s {[\Cl]_s}^2}{Na_k K_k {[\Cl]_k}^2}\right)
\end{equation}
%
Flux through the sodium potassium pump   (times the \gls{AC} volume-area ratio $R_k$; in \uMmps): 
\begin{equation} \label{eq:J_NaK_s}
J_{\text{NaK},k}=J_{\text{NaK,max}}\frac{{Na_k}^{1.5}}{{Na_k}^{1.5}+{K_{\text{Na},k}}^{1.5}}\frac{[\K]_s}{[\K]_s+K_{\text{K},s}}
\end{equation}
%
\gls{K} flux through the BK channel  (times the \gls{AC} volume-area ratio $R_k$; in \uMmps): 
\begin{equation} \label{eq:J_BK}
J_{\text{BK},k}=\frac{g_{\text{BK},k}}{F}w_k \left(v_k - E_{\text{BK},k} \right)
\end{equation}
%
%
%
\begin{table}[h!]
\centering
\begin{tabular}{ p{0.09\linewidth}  >{\footnotesize} p{0.5\linewidth}  >{\footnotesize} p{0.27\linewidth} >{\footnotesize} p{0.03\linewidth} }
\hline	
$F$ 			& Faraday's constant														& 9.649$\times$10$^4$ \Cmol 	& \\
$R_{\text{gas}}$ 			& Gas constant 															& 8.315 \JmolK		& \\
$T$ 	    	& Temperature 															& 300 \Kelvin		& \\
$g_{K_{k}}$ 	& Specific ion conductance of potassium 								& 40$\times$10$^3$ \perOhmm 		& \cite{Ostby2009}  \\
$g_{\text{Na},k}$ 		& Specific ion conductance of sodium 									& 1.314$\times$10$^3$  \perOhmm 	& \cite{Ostby2009}  \\
$g_{\text{NBC},k}$ 	& Specific ion conductance of the NBC cotransporter						& 7.57$\times$10$^2$ \perOhmm 	& \cite{Ostby2009}  \\
$g_{\text{KCC1},k}$ 	& Specific ion conductance of the KCC1 cotransporter					& 10 \perOhmm 	& \cite{Ostby2009}  \\
$g_{\text{NKCC1},k}$ 	& Specific ion conductance of the NKCC1 cotransporter	 				& 55.4 \perOhmm 	& \cite{Ostby2009}  \\
$J_{\text{NaK,max}}$ & Maximum flux through the NaKATPase pump						     	& 1.42$\times$10$^{-3}$ \uMms 	& \cite{Ostby2009}  \\
$G_{\text{BK},k}$ 		& Potassium conductance of the BK channel							& 4.3$\times$10$^3$   pS & \cite{GonzalezFernandez1994}  \\
%$A_{ef,k}$		& Membrane area														    & 3.7 $\times$ 10$^{-9}$ & \cite{GonzalezFernandez1994} \\
$C_{input}$  & Block function to switch the channel on and off &  0 ; 1 [-] 			&  \\
$K_{\text{Na},k}$  & Michaelis-Menten constant   &  10$^4$ \uM &   \\
$K_{\text{K},s}$  & Michaelis-Menten constant   &  1.5 $ \times $ 10$^3$ \uM &   \\
$C_{\text{unit}}$ & Unit converting factor   & 10$^{3}$ & M.E. \\
\hline
\end{tabular}
\end{table}


Specific ion conductance of the BK channel (\perOhmm):%  \cite{GonzalezFernandez1994}  
\begin{equation}
	g_{\text{BK},k} = \frac{G_{\text{BK},k} \times 10^{-12}}{A_{k}} = 1.16 \times 10^3 \text{ \perOhmm}
\end{equation}

\subsubsection*{Additional Equations}
\paragraph{Synaptic Cleft}~\\
%
\gls{Cl} concentration  (times the SC volume-area ratio $R_s$; in \uMm): 
\begin{equation} \label{eq:ClEx}
N_{\text{Cl},s}= N_{\text{Na},s}+N_{\text{K},s}-N_{ HCO_{3,s}}\,; \text{~~~} [\Cl]_s = \frac{N_{\text{Cl},s}}{R_s}
\end{equation}

\paragraph{Astrocyte}~\\
%
Membrane voltage of the \gls{AC} (V):
\begin{equation} \label{eq:v_k}
v_k = \frac{g_{\text{Na},k} E_{\text{Na},k} + g_{\text{K},k} E_{\text{K},k} + g_{\text{Cl},k}E_{\text{Cl},k}+g_{\text{NBC},k}E_{\text{NBC},k} + g_{\text{BK},k}w_kE_{\text{BK},k} -J_{\text{NaK},k}F C_{\text{unit}} } { g_{\text{Na},k}+g_{\text{K},k} + g_{\text{Cl},k} + g_{\text{NBC},k} + g_{\text{BK},k}w_k }
\end{equation}
%
Nernst potential for the potassium channel (in mV):
\begin{equation} \label{eq:E_K}
E_{\text{K},k}=\frac{R_{\text{gas}}T}{z_K F}ln\left( \frac{[\K]_s}{[\K]_k}\right) 
\end{equation}
%
Nernst potential for the sodium channel (in mV):
\begin{equation} \label{eq:E_Na}
E_{\text{Na},k}=\frac{R_{\text{gas}}T}{z_{\text{Na}} F}ln\left( \frac{Na_s}{Na_k}\right) 
\end{equation}
%
Nernst potential for the chloride channel (in mV):
\begin{equation} \label{eq:E_Cl}
E_{\text{Cl},k} = \frac{R_{\text{gas}}T}{z_{\text{Cl}} F}ln\left( \frac{[\Cl]_s}{[\Cl]_k}\right) 
\end{equation}
%
Nernst potential for the NBC channel (in mV):
\begin{equation} \label{eq:E_NBC}
E_{\text{NBC},k} = \frac{R_{\text{gas}}T}{z_{NBC} F}ln\left( \frac{Na_s {HCO_{3,s}}^2}{Na_k {HCO_{3,k}}^2}\right) 
\end{equation}
Nernst potential for the BK channel (in mV):
\begin{equation} \label{eq:E_BK}
E_{\text{BK},k} = \frac{R_{\text{gas}} T}{z_K F}\ln\left( \frac{[\K]_p}{[\K]_k}\right) 
\end{equation}
%
Equilibrium state BK-channel (-):
\begin{equation} \label{eq:winf}
w_{\infty} = 0.5 \left(1+\tanh\left(\frac{v_{k}+v_{6} }{v_{4}} \right)  \right) 
\end{equation}
%
Time constant associated with the opening of BK channels	 (in \pers):
\begin{equation} \label{eq:phin}
\phi_{w} = \psi_{w}\cosh\left( \frac{v_{k}+v_{6}}{2v_{4}}\right) 
\end{equation}

\begin{table}[h!]
\centering
\begin{tabular}{ p{0.09\linewidth}  >{\footnotesize} p{0.6\linewidth}  >{\footnotesize} p{0.17\linewidth} >{\footnotesize} p{0.03\linewidth} }
\hline
$g_{\text{Cl},k}$ 		& Specific ion conductance of chloride 									& 0.879 \perOhmm & \cite{Ostby2009}  \\
$z_K$			& Valence of a potassium ion										& 1   & \\ 
$z_{\text{Na}}$			& Valence of a sodium ion											& 1   & \\ 
$z_{\text{Cl}}$			& Valence of a chloride ion											& -1  & \\ 
$z_{NBC}$ 		& Effective valence of the NBC cotransporter complex 				& -1 & \\
$v_{6}$			& Voltage associated with the opening of half the population		& 22 mV or V???  & \cite{GonzalezFernandez1994}  \\
$v_{4}$			& A measure of the spread of the distribution of the open probability of the BK channel	& 14.5 mV or V???  &  \cite{GonzalezFernandez1994}  
\\
$ \psi_{w}$    	& A characteristic time for the open probability of the BK channel		& 2.664 \pers & \cite{GonzalezFernandez1994} \\
\hline
\end{tabular}
\end{table}

\subsection{The Smooth Muscle Cell and Endothelial Cell Model}\label{sec:EqSMCEC}

\subsubsection*{Conservation Equations}
\paragraph{Smooth muscle cell}~\\
%
Cytosolic [\gls{Ca}] in the \gls{SMC} (in \uM):
\begin{equation}\label{eq:ci}
\begin{split}
\dfrac{\mathrm{d}\CaConsc}{\mathrm{d}t} = J_{\text{IP}_3,i} - J_{\text{upt},i} + J_{\text{CICR}_i} - J_{\text{extr},i} +  J_{\text{leak},i}\dots \\
 - J_{\text{VOCC},i} + J_{\text{Na/Ca},i}  + 0.1J_{\text{stretch},i} + J_{Ca^{2+}-coupling_{i}}^{SMC-EC}
\end{split} 
\end{equation}
%
[Ca$^{2+}$] in the \gls{SR} of the \gls{SMC} (in \uM):
\begin{equation} \label{eq:si}
\dfrac{\mathrm{d}\CaConse}{\mathrm{d}t} =  J_{\text{upt},i} - J_{\text{CICR}_i} - J_{\text{leak},i}
\end{equation}
%
Membrane potential of the \gls{SMC} (in \mV):
\begin{equation} \label{eq:vi}
\begin{split}
\dfrac{\mathrm{d}v_{i}}{\mathrm{d}t} = \gamma_{i}( -J_{\text{Na/K},i} - J_{\text{Cl},i} - 2J_{\text{VOCC},i}- J_{\text{Na/Ca},i} - J_{\text{K},i} \dots \\
- J_{\text{stretch},i} - J_{\text{KIR},i} ) +V^{SMC-EC}_{coupling_{i}}
\end{split}
\end{equation}
%
Open state probability of calcium-activated potassium channels (dim.less):
\begin{equation} \label{eq:dwidt}
\dfrac{\mathrm{d}w_{i}}{\mathrm{d}t} =  \lambda_{i} \left( K_{act_{i}} - w_{i} \right)
\end{equation}
%
\gls{IP3} concentration om the \gls{SMC} (in \uM):
\begin{equation} \label{eq:dIidt}
\dfrac{\mathrm{d}\IP _{i}}{\mathrm{d}t} = J^{SMC-EC}_{IP_{3}-coupling_{i}} - J_{\text{degr},i}
\end{equation}
%
\gls{K} concentration in the \gls{SMC} (in \uM):
\begin{equation} \label{eq:dkidt}
\dfrac{\mathrm{d} [\K]_i}{\mathrm{d}t}  = J_{\text{Na/K},i}  - J_{\text{KIR},i} - J_{\text{K},i}
\end{equation}

\begin{table}[h!]
\centering
\begin{tabular}{ p{0.09\linewidth}  >{\footnotesize} p{0.5\linewidth}  >{\footnotesize} p{0.27\linewidth} >{\footnotesize} p{0.03\linewidth} }
\hline
$\gamma_{i}$				& Change in membrane potential by a scaling factor					& 1970 \mVpuM	& \cite{Koenigsberger2006} \\
$\lambda_{i} $				& Rate constant for opening											& 45.0 \pers 	& \cite{Koenigsberger2006} \\
%$\CaConsc$      		& Cytololic [Ca$^{2+}$] in the SMC    								& var. \uM		& - \\
\hline
\end{tabular}
\label{tab:dcidt}
\end{table}

\paragraph{Endothelial cell}~\\
%
Cytosolic \gls{Ca} concentration in the \gls{EC} (in \uM):
\begin{equation} \label{eq:cj}
\begin{split}
\dfrac{\mathrm{d}\CaConec}{\mathrm{d}t} = J_{\text{IP}_3,j} - J_{\text{upt},j} + J_{CICR_{j}} - J_{\text{extr},j}\dots \\
 + J_{\text{leak},j} + J_{cation_{j}} + J_{0_{j}} + J_{\text{stretch},j} - J_{Ca^{2+}-coupling_{j}}^{SMC-EC}
\end{split}
\end{equation}
%
\gls{Ca} concentration in the \gls{ER} in the \gls{EC} (in \uM): %copied from SMC
\begin{equation} \label{eq:sj}
\dfrac{\mathrm{d}\CaConee}{\mathrm{d}t} =  J_{\text{upt},j} - J_{CICR_{j}} - J_{\text{leak},j}
\end{equation}
%
Membrane potential of the \gls{EC} (in \mV):
\begin{equation} \label{eq:dvjdt}
\dfrac{\mathrm{d}v_{j}}{\mathrm{d}t} =-\frac{1}{C_{m_{j}}} ( J_{K_{j}}+J_{R_{j}}) + V^{SMC-EC}_{coupling_{j}}
\end{equation}
%
\gls{IP3} concentration of the \gls{EC} (in \uM):
\begin{equation} \label{eq:dIjdt}
\dfrac{\mathrm{d}\IP_{j}}{\mathrm{d}t} =  J_{\text{EC,IP}_3}- J_{\text{degr},j}  - J^{SMC-EC}_{IP_{3}-coupling_{j}}
\end{equation}

\begin{table}[h!]
\centering
\begin{tabular}{ p{0.09\linewidth}  >{\footnotesize} p{0.5\linewidth}  >{\footnotesize} p{0.27\linewidth} >{\footnotesize} p{0.03\linewidth} }
\hline
 $C_{m_{j}}$				& Membrane capacitance												& 25.8  \pF		& \cite{Koenigsberger2006} \\
 $ J_{PLC} $  & PLC / \gls{IP3} production rate & 0.18 or 0.4 \uMps & \cite{Koenigsberger2006}  \\
 $J_{0_{j}}$ & Constant Ca2+ leak term (influx) & 0.029 \uMps & \cite{Koenigsberger2006} \\ 
\hline
\end{tabular}
\label{tab:JSRuptakei}
\end{table}
%\\

\subsubsection*{Fluxes}
%
\paragraph{Smooth muscle cell}~\\
%
Release of calcium from IP$_{3}$ sensitive stores in the SMC (in \uMps):
\begin{equation} \label{eq:IP3i}
J_{\text{IP}_3,i} = F_{i}\frac{\IP_{i}^{2}}{K_{ri}^{2}+\IP_{i}^{2}}
\end{equation}
%
\begin{table}[h!]
\centering
\begin{tabular}{ p{0.09\linewidth}  >{\footnotesize} p{0.5\linewidth}  >{\footnotesize} p{0.27\linewidth} >{\footnotesize} p{0.03\linewidth} }
\hline
 $F_{i}$      			& Maximal rate of activation-dependent calcium influx			& 0.23 \uMps				& \cite{Koenigsberger2006} \\
$K_{ri}$				& Half-saturation constant for agonist-dependent calcium entry	& 1 \uM					& \cite{Koenigsberger2006} \\
\hline
\end{tabular}
\label{tab:IP3i}
\end{table}
\\
%
Uptake of calcium into the sarcoplasmic reticulum (in \uMs):
\begin{equation} \label{eq:JSRuptakei}
J_{\text{upt},i} = B_{i}\frac{\CaConsc^{2}}{c_{bi}^{2}+\CaConsc^{2}}
\end{equation}
%
\begin{table}[h!]
\centering
\begin{tabular}{ p{0.09\linewidth}  >{\footnotesize} p{0.5\linewidth}  >{\footnotesize} p{0.27\linewidth} >{\footnotesize} p{0.03\linewidth} }
\hline
$B_{i}$      			& SR uptake rate constant							& 2.025 \uMs				& \cite{Koenigsberger2006} \\
$c_{bi}$				& Half-point of the SR ATPase activation sigmoidal 	& 1.0 \uM					& \cite{Koenigsberger2006} \\
\hline
\end{tabular}
\label{tab:JSRuptakei}
\end{table}
\\
%
Calcium-induced calcium release (CICR; in \uMs):
\begin{equation} \label{eq:JCICRi}
J_{\text{CICR}_i} = C_{i}\frac{\CaConse^{2}}{s_{ci}^{2}+\CaConse^{2}}    \frac{\CaConsc^{4}}{c_{ci}^{4}+\CaConsc^{4}}
\end{equation}
%
\begin{table}[h!]
\centering
\begin{tabular}{ p{0.09\linewidth}  >{\footnotesize} p{0.5\linewidth}  >{\footnotesize} p{0.27\linewidth} >{\footnotesize} p{0.03\linewidth} }
\hline
$C_{i}$      			& CICR rate constant									& 55 \uMs		& \cite{Koenigsberger2006} \\
$s_{ci}$				& Half-point of the CICR Ca$^{2+}$ efflux sigmoidal			& 2.0 \uM		& \cite{Koenigsberger2006} \\
$c_{ci}$				& Half-point of the CICR activation sigmoidal			& 0.9 \uM		& \cite{Koenigsberger2006} \\
\hline
\end{tabular}
\label{tab:JCICRi}
\end{table}
\\
%
Calcium extrusion by Ca$^{2+}$-ATPase pumps (in \uMs):
\begin{equation} \label{eq:Jextrusioni}
J_{\text{extr},i} = D_{i}\CaConsc   \left( 1+ \frac{v_{i}-v_{d}}{R_{di}}\right)
\end{equation}
%
\begin{table}[h!]
\centering
\begin{tabular}{ p{0.09\linewidth}  >{\footnotesize} p{0.5\linewidth}  >{\footnotesize} p{0.27\linewidth} >{\footnotesize} p{0.03\linewidth} }
\hline
$D_{i}$      			& Rate constant for Ca$^{2+}$ extrusion by the ATPase pump		 & 0.24	\pers			& \cite{Koenigsberger2006} \\
$v_{d}$					& Intercept of voltage dependence of extrusion ATPase			 & -100.0 \mV			& \cite{Koenigsberger2006} \\
$R_{di}$				& Slope of voltage dependence of extrusion ATPase.				 & 250.0 \mV			& \cite{Koenigsberger2006} \\
\hline
\end{tabular}
\label{tab:Jextrusioni}
\end{table}
\\
%
Leak current from the SR (in \uMs):
\begin{equation} \label{eq:JSRleaki}
J_{\text{leak},i} = L_{i}\CaConse
\end{equation}
\begin{table}[h!]
\centering
\begin{tabular}{ p{0.09\linewidth}  >{\footnotesize} p{0.5\linewidth}  >{\footnotesize} p{0.27\linewidth} >{\footnotesize} p{0.03\linewidth} }
\hline
$L_{i}$      			& Leak from SR rate constant						 & 0.025 \pers				& \cite{Koenigsberger2006} \\
\hline
\end{tabular}
\label{tab:Jleaki}
\end{table}
\\

Calcium influx through VOCCs (in \uMs): 
\begin{equation} \label{eq:JVOCCi}
J_{\text{VOCC},i} = G_{\text{Ca},i} \frac{v_{i}-v_{\text{Ca}_1,i}}     {1+ \exp(-\left[ \left(  v_{i}-v_{\text{Ca}_2,i}\right) /R_{\text{Ca},i}      \right] )}
\end{equation}
\begin{table}[h!]
\centering
\begin{tabular}{ p{0.09\linewidth}  >{\footnotesize} p{0.5\linewidth}  >{\footnotesize} p{0.27\linewidth} >{\footnotesize} p{0.03\linewidth} }
\hline
$G_{\text{Ca},i}$      	& Whole-cell conductance for VOCCs	 					& 1.29$\times$10$^{-3}$  \uMpmVs					& \cite{Koenigsberger2006} \\
$v_{\text{Ca}_1,i}$   & Reversal potential for VOCCs	 						& 100.0 \mV							& \cite{Koenigsberger2006} \\
$v_{\text{Ca}_2,i}$  	& Half-point of the VOCC activation sigmoidal		 	& -24.0 \mV							& \cite{Koenigsberger2006} \\
$R_{\text{Ca},i}$      	& Maximum slope of the VOCC	activation sigmoidal		& 8.5 \mV							& \cite{Koenigsberger2006} \\
\hline
\end{tabular}
\label{tab:JVOCCi}
\end{table}
\newpage
Flux of calcium exchanging with sodium in the Na$^{+}$Ca$^{2+}$ exchange (in \uMs): 
\begin{equation} \label{eq:JNaCai}
J_{\text{Na/Ca},i} = G_{\text{Na/Ca},i} \frac{\CaConsc}     {\CaConsc + c_{\text{Na/Ca},i}} \left( v_{i}-v_{\text{Na/Ca},i} \right)
\end{equation}
%
\begin{table}[h!]
\centering
\begin{tabular}{ p{0.09\linewidth}  >{\footnotesize} p{0.5\linewidth}  >{\footnotesize} p{0.27\linewidth} >{\footnotesize} p{0.03\linewidth} }
\hline
$G_{\text{Na/Ca},i}$   	& Whole-cell conductance for Na$^{+}$/Ca$^{2+}$ exchange			 		 & 3.16$\times$10$^{-3}$ \uMpmVs	& \cite{Koenigsberger2006} \\
$c_{\text{Na/Ca},i}$   	& Half-point for activation of Na$^{+}$/Ca$^{2+}$ exchange by Ca$^{2+}$		 & 0.5 \uM			& \cite{Koenigsberger2006} \\
$v_{\text{Na/Ca},i}$   	& Reversal potential for the Na$^{+}$/Ca$^{2+}$ exchanger					 & -30.0 \mV		& \cite{Koenigsberger2006} \\
\hline
\end{tabular}
\label{tab:JNaCai}
\end{table}
\\
%
Calcium flux through the stretch-activated channels in the SMC (in \uMs): 
\begin{equation} \label{eq:Jstretchi}
\begin{split}
J_{\text{stretch},i}= \frac{G_{\text{stretch}}}{1+ \exp\left(-\alpha_{\text{stretch}}  \left(  \frac{\Delta pR}{h} -\sigma_{0}   \right) \right)}  \left(  v_{i}-E_{\text{SAC}}   \right) 
\end{split}
\end{equation}
%
\begin{table}[h!]
\centering
\begin{tabular}{ p{0.09\linewidth}  >{\footnotesize} p{0.5\linewidth}  >{\footnotesize} p{0.27\linewidth} >{\footnotesize} p{0.03\linewidth} }
\hline
$G_{\text{stretch}}$      		& Whole cell conductance for SACs						& 6.1$\times$10$^{-3}$ \uMpmVs	&\cite{Koenigsberger2006} \\
$\alpha_{\text{stretch}}$      & Slope of stress dependence of the SAC activation sigmoidal	& 7.4$\times$10$^{-3}$ \pmmHg	&\cite{Koenigsberger2006} \\
$ \Delta p $			& Pressure difference										& 30 \mmHg			& ME \\
$\sigma_{0}$      		& Half-point of the SAC activation sigmoidal				& 500 \mmHg			&\cite{Koenigsberger2006} \\
$E_{\text{SAC}}$      			& Reversal potential for SACs							& -18 \mV			&\cite{Koenigsberger2006} \\
\hline
\end{tabular}
\label{tab:Jstretchi}
\end{table}
\\
%
Flux through the sodium potassium pump (in \uMs): 
\begin{equation} \label{eq:J_NaK_i}
J_{\text{Na/K},i}= F_{\text{Na/K},i}
\end{equation}
%
\begin{table}[h!]
\centering
\begin{tabular}{ p{0.09\linewidth}  >{\footnotesize} p{0.5\linewidth}  >{\footnotesize} p{0.27\linewidth} >{\footnotesize} p{0.03\linewidth} }
\hline
$F_{\text{Na/K},i}$      			& Rate of the potassium influx by the sodium potassium pump 		& 4.32$\times$10$^{-2}$ \uMps 	&\cite{Koenigsberger2006} \\
\hline
\end{tabular}
\label{tab:JCli}
\end{table}
\\
Chloride flux through the chloride channel (in \uMs):
\begin{equation} \label{eq:JCli}
J_{\text{Cl},i} = G_{\text{Cl},i} \left(  v_{i} - v_{\text{Cl},i}  \right) 
\end{equation}
%
\begin{table}[h!]
\centering
\begin{tabular}{ p{0.09\linewidth}  >{\footnotesize} p{0.5\linewidth}  >{\footnotesize} p{0.27\linewidth} >{\footnotesize} p{0.03\linewidth} }
\hline
$G_{\text{Cl},i}$      			& Whole-cell conductance for Cl$^{-}$ current		& 1.34$\times$10$^{-3}$ \uMpmVs	&\cite{Koenigsberger2006} \\
$v_{\text{Cl},i}$      			& Reversal potential for Cl$^{-}$ channels.			& -25.0 \mV			&\cite{Koenigsberger2006} \\
\hline
\end{tabular}
\label{tab:JCli}
\end{table}
\\
%
Potassium flux through potassium channel (in \uMs):
\begin{equation} \label{eq:JKi}
J_{\text{K},i}= G_{\text{K},i} w_{i} \left(  v_{i} - v_{K_i}  \right) 
\end{equation}
%
\begin{table}[h!]
\centering
\begin{tabular}{ p{0.09\linewidth}  >{\footnotesize} p{0.5\linewidth}  >{\footnotesize} p{0.27\linewidth} >{\footnotesize} p{0.03\linewidth} }
\hline
$G_{\text{K},i}$      			& Whole-cell conductance for K$^{+}$ efflux.			& 4.46$\times$10$^{-3}$ \uMpmVs	&\cite{Koenigsberger2006} \\
$v_{K_i}$      			& Nernst potential										& -94 \mV	&\cite{Koenigsberger2006} \\
\hline
\end{tabular}
\label{tab:JKi}
\end{table}
\\
Flux through KIR channels in the SMC (in \uMs): 
\begin{equation} \label{eq:JKIRi}
J_{\text{KIR},i} =  \frac{F_{\text{KIR},i} g_{\text{KIR},i}}{\gamma_{i}}( v_{i} - v_{\text{KIR},i})
\end{equation}
\begin{table}[h!]
\centering
\begin{tabular}{ p{0.09\linewidth}  >{\footnotesize} p{0.5\linewidth}  >{\footnotesize} p{0.27\linewidth} >{\footnotesize} p{0.03\linewidth} }
\hline
$ F_{\text{KIR},i} $ & Scaling factor of potassium efflux through the KIR channel & 750 mV~\textmu M$^{-1}$ & \cite{GonzalezFernandez1994} \\
\hline
\end{tabular}
\label{tab:JCli}
\end{table}
\\
IP$_{3}$ degradation (in \uMs): 
\begin{equation} \label{eq:Jdegradi}
J_{\text{degr},i}= k_{\text{d},i}\IP_i
\end{equation}
\begin{table}[h!]
\centering
\begin{tabular}{ p{0.09\linewidth}  >{\footnotesize} p{0.5\linewidth}  >{\footnotesize} p{0.27\linewidth} >{\footnotesize} p{0.03\linewidth} }
\hline
$k_{\text{d},i}$      			& Rate constant of IP$_{3}$ degradation	& 0.1 \pers	&\cite{Koenigsberger2006} \\
\hline
\end{tabular}
\label{tab:Jdegradi}
\end{table}
\paragraph{Endothelial cell}~\\
\\
%
Release of calcium from IP$_{3}$-sensitive stores in the EC (in \uMps):
\begin{equation} \label{eq:JIP3j}
J_{\text{IP}_3,j} = F_{j}\frac{\IP_{j}^{2}}{K_{rj}^{2}+\IP_{j}^{2}}
\end{equation}
\begin{table}[h!]
\centering
\begin{tabular}{ p{0.09\linewidth}  >{\footnotesize} p{0.5\linewidth}  >{\footnotesize} p{0.27\linewidth} >{\footnotesize} p{0.03\linewidth} }
\hline
 $F_{j}$      			& Maximal rate of activation-dependent calcium influx			& 0.23 \uMps				& \cite{Koenigsberger2006} \\
$K_{rj}$				& Half-saturation constant for agonist-dependent calcium entry	& 1 \uM					& \cite{Koenigsberger2006} \\
\hline
\end{tabular}
\label{tab:IP3j}
\end{table}
\\
%
Uptake of calcium into the endoplasmic reticulum (in \uMs):
\begin{equation} \label{eq:JERuptakej}
J_{\text{upt},j} = B_{j}\frac{\CaConec^{2}}{c_{bj}^{2}+\CaConec^{2}}
\end{equation}
%
\begin{table}[h!]
\centering
\begin{tabular}{ p{0.09\linewidth}  >{\footnotesize} p{0.5\linewidth}  >{\footnotesize} p{0.27\linewidth} >{\footnotesize} p{0.03\linewidth} }
\hline
$B_{j}$      			& ER uptake rate constant							& 0.5 \uMs				& \cite{Koenigsberger2006} \\
$c_{bj}$				& Half-point of the SR ATPase activation sigmoidal 	& 1.0 \uM					& \cite{Koenigsberger2006} \\
\hline
\end{tabular}
\label{tab:JERuptakej}
\end{table}
\\
%
Calcium-induced calcium release (CICR; in \uMs):
\begin{equation} \label{eq:JCICRJ}
J_{CICR_{j}} = C_{j}\frac{\CaConee^{2}}{s_{cj}^{2}+\CaConee^{2}}    \frac{\CaConec^{4}}{c_{cj}^{4}+\CaConec^{4}}
\end{equation}
%
\begin{table}[h!]
\centering
\begin{tabular}{ p{0.09\linewidth}  >{\footnotesize} p{0.5\linewidth}  >{\footnotesize} p{0.27\linewidth} >{\footnotesize} p{0.03\linewidth} }
\hline
$C_{j}$      			& CICR rate constant									& 5 \uMs		& \cite{Koenigsberger2006} \\
$s_{cj}$				& Half-point of the CICR Ca$^{2+}$ efflux sigmoidal			& 2.0 \uM		& \cite{Koenigsberger2006} \\
$c_{cj}$				& Half-point of the CICR activation sigmoidal			& 0.9 \uM		& \cite{Koenigsberger2006} \\
\hline
\end{tabular}
\label{tab:JCICRj}
\end{table}
\\
Calcium extrusion by Ca$^{2+}$-ATPase pumps (in \uMs):
\begin{equation} \label{eq:Jextrusionj}
J_{\text{extr},j} = D_{j}\CaConec 
\end{equation}
%
%
\begin{table}[h!]
\centering
\begin{tabular}{ p{0.09\linewidth}  >{\footnotesize} p{0.5\linewidth}  >{\footnotesize} p{0.27\linewidth} >{\footnotesize} p{0.03\linewidth} }
\hline
$D_{j}$      			& Rate constant for Ca$^{2+}$ extrusion by the ATPase pump		 & 0.24	\pers			& \cite{Koenigsberger2005} \\
\hline
\end{tabular}
\label{tab:Jextrusionj}
\end{table}
\\ 
Calcium flux through the stretch-activated channels in the EC (in \uMs): 
\begin{equation} \label{eq:Jstretchj}
J_{\text{stretch},j}= \frac{G_{\text{stretch}}}{1+ e^{-\alpha_{\text{stretch}}  \left(  \sigma -\sigma_{0}   \right) }}  \left(  v_{j}-E_{\text{SAC}}   \right) \\
= \frac{G_{\text{stretch}}}{1+ e^{-\alpha_{\text{stretch}}  \left(  \frac{\Delta pR}{h} -\sigma_{0}   \right) }}  \left(  v_{j}-E_{\text{SAC}}   \right) 
\end{equation}
%
\begin{table}[h!]
\centering
\begin{tabular}{ p{0.09\linewidth}  >{\footnotesize} p{0.5\linewidth}  >{\footnotesize} p{0.27\linewidth} >{\footnotesize} p{0.03\linewidth} }
\hline
$G_{\text{stretch}}$      		& The whole cell conductance for SACs						& 6.1$\times$10$^{-3}$ \uMpmVs	&\cite{Koenigsberger2006} \\
$\alpha_{\text{stretch}}$      & Slope of stress dependence of the SAC activation sigmoidal	& 7.4$\times$10$^{-3}$ \pmmHg	&\cite{Koenigsberger2006} \\
$ \Delta p $			& Pressure difference										& 30 \mmHg			& ME \\
$\sigma_{0}$      		& Half-point of the SAC activation sigmoidal				& 500 \mmHg			&\cite{Koenigsberger2006} \\
$E_{\text{SAC}}$      			& The reversal potential for SACs							& -18 \mV			&\cite{Koenigsberger2006} \\
\hline
\end{tabular}
\label{tab:Jstretchj}
\end{table}
\todo{table ist doppelt?!}
\\
%
Leak current from the ER (in \uMs):
\begin{equation} \label{eq:JERleakj}
J_{\text{leak},j} = L_{j}\CaConee
\end{equation}
%
\begin{table}[h!]
\centering
\begin{tabular}{ p{0.09\linewidth}  >{\footnotesize} p{0.5\linewidth}  >{\footnotesize} p{0.27\linewidth} >{\footnotesize} p{0.03\linewidth} }
\hline
$L_{j}$      			& Rate constant for Ca$^{2+}$ leak from the ER 		 & 0.025	\pers			& \cite{Koenigsberger2006} \\
\hline
\end{tabular}
\label{tab:JKj}
\end{table}
\\
%
Calcium influx through nonselective cation channels (in \uMs):
\begin{equation} \label{eq:Jcationj}
J_{cation_{j}} = G_{cat_{j}} (E_{Ca_{j}} - v_{j}) \frac{1}{2} \left(   1+ \tanh \left(  \frac{\mathrm{log}_{10} \CaConec - m_{3_{cat_{j}}} }    {m_{4_{cat_{j}}}}   \right)      \right) 
\end{equation}
%
%
\begin{table}[h!]
\centering
\begin{tabular}{ p{0.09\linewidth}  >{\footnotesize} p{0.5\linewidth}  >{\footnotesize} p{0.27\linewidth} >{\footnotesize} p{0.03\linewidth} }
\hline
$G_{cat j}$      		& Whole-cell cation channel conductivity						 	& 6.6$\times$10$^{-4}$ \uMpmVs	& \cite{Koenigsberger2006} \\
$E_{Caj}$      			& Ca$^{2+}$ equilibrium potential								 	& 50 \mV		& \cite{Koenigsberger2006} \\

$m_{3_{catj}}$      	& Model constant				 	& -0.18 \uM		& \cite{Koenigsberger2006} \\
$m_{4_{catj}}$      	& Model constant					& 0.37  \uM		& \cite{Koenigsberger2006} \\
\hline
\end{tabular}
\label{tab:Jcationj}
\end{table}
\\
%
Potassium efflux through the $J_{BK_{Caj}}$ channel and the $J_{SK_{Caj}}$ channel (in \uMs):
\begin{equation} \label{eq:JKj}
J_{K_{j}} = G_{totj} (v_{j}-v_{Kj}) \left(   J_{BK_{Caj}} + J_{SK_{Caj}} \right) 
\end{equation}
%
%
\begin{table}[h!]
\centering
\begin{tabular}{ p{0.09\linewidth}  >{\footnotesize} p{0.5\linewidth}  >{\footnotesize} p{0.27\linewidth} >{\footnotesize} p{0.03\linewidth} }
\hline
$G_{totj}$      		& Total potassium channel conductivity.						 		& 6927 \pS		& \cite{Koenigsberger2006} \\
$v_{Kj}$      			& K$^{+}$ equilibrium potential					 			 		& -80.0 \mV		& \cite{Koenigsberger2006} \\
\hline
\end{tabular}
\label{tab:JKj}
\end{table}
\\
%
Potassium efflux through the $J_{BK_{Caj}}$ channel (in \uMs):
\begin{equation} \label{eq:JBKCAj}
J_{BK_{Caj}} = 0.2 \left(   1+ \tanh \left(   \frac{   (\mathrm{log}_{10} \CaConec - c) (v_{j}-b_{j}) - a_{1j}  }   { m_{3bj} ( v_{j} + a_{2j} (\mathrm{log}_{10} \CaConec -c )-b_{j} )^{2} + m_{4bj} }  \right)     \right)  
\end{equation}
%
Potassium efflux through the $J_{SK_{Caj}}$ channel (in \uMs):
\begin{equation} \label{eq:JSKCaj}
J_{SK_{Caj}} = 0.3\left( 1+ \tanh \left(  \frac{   \mathrm{log}_{10} \CaConec -m_{3sj}  } {m_{4sj}}  \right)      \right) 
\end{equation}
%
\begin{table}[h!]
\centering
\begin{tabular}{ p{0.09\linewidth}  >{\footnotesize} p{0.5\linewidth}  >{\footnotesize} p{0.27\linewidth} >{\footnotesize} p{0.03\linewidth} }
\hline
$c$      				& Model constant, further explanation see reference					& -0.4 \uM			& \cite{Koenigsberger2006} \\
$b_{j}$      			& Model constant, further explanation see reference					& -80.8 \mV		& \cite{Koenigsberger2006} \\
$a_{1j}$      			& Model constant, further explanation see reference					& 53.3 \uMkeermV	& \cite{Koenigsberger2006} \\
$a_{2j}$      			& Model constant, further explanation see reference					& 53.3 \mVpuM		& \cite{Koenigsberger2006} \\
$m_{3bj}$      			& Model constant, further explanation see reference					& 1.32$\times$10$^{-3}$ \uMpmV	& \cite{Koenigsberger2006} \\
$m_{4bj}$      			& Model constant, further explanation see reference					& 0.30	\uMkeermV	& \cite{Koenigsberger2006} \\
$m_{3sj}$      			& Model constant, further explanation see reference					& -0.28 \uM		& \cite{Koenigsberger2006} \\
$m_{4sj}$      			& Model constant, further explanation see reference					& 0.389 \uM		& \cite{Koenigsberger2006} \\
\hline
\end{tabular}
\label{tab:JBKCAj}
\end{table}
\\
%
Residual current regrouping chloride and sodium current flux (in \uMs):
\begin{equation} \label{eq:JRj}
J_{R_{j}} = G_{R_{j}} ( v_{j} - v_{\text{rest},j}  )
\end{equation}
%
\begin{table}[h!]
\centering
\begin{tabular}{ p{0.09\linewidth}  >{\footnotesize} p{0.5\linewidth}  >{\footnotesize} p{0.27\linewidth} >{\footnotesize} p{0.03\linewidth} }
\hline
$G_{R_{j}}$      		& Residual current conductivity										& 955 \pS			& \cite{Koenigsberger2006} \\
$v_{\text{rest},j}$      		& Membrane resting potential						 				& -31.1 \mV		& \cite{Koenigsberger2006} \\
\hline
\end{tabular}
\label{tab:JRj}
\end{table}
\\
%
IP$_{3}$ degradation (in \uMs):  
\begin{equation} \label{eq:Jdegradj}
J_{\text{degr},j}= k_{\text{d},j} \IP_{j}
\end{equation}
%
\begin{table}[h!]
\centering
\begin{tabular}{ p{0.09\linewidth}  >{\footnotesize} p{0.5\linewidth}  >{\footnotesize} p{0.27\linewidth} >{\footnotesize} p{0.03\linewidth} }
\hline
$k_{\text{d},j}$      			& Rate constant of IP$_{3}$ degradation						 		& 0.1 \pers		& \cite{Koenigsberger2006} \\
\hline
\end{tabular}
\label{tab:Jdegradj}
\end{table}
\\
%
%
\subsubsection*{Coupling}~\\
%
Heterocellular electrical coupling between SMCs and ECs (in \mVs):
\begin{equation} \label{eq:Vcouplingi}
V_{coupling_{i}}^{SMC-EC}= -G_{coup}(v_{i}-v_{j})
\end{equation}
%
Heterocellular IP$_{3}$ coupling between SMCs and ECs (in \uMs):
\begin{equation} \label{eq:JIP3couplingi}
J_{IP_{3}-coupling_{i}}^{SMC-EC}= -P_{IP_{3}}(\IP_{i}-\IP_{j})
\end{equation}
%
Calcium coupling with EC (in \uMs):
\begin{equation} \label{eq:JCAcouplingi}
J_{Ca^{2+}-coupling_{i}}^{SMC-EC}= -P_{Ca^{2+}}(\CaConsc-\CaConec)
\end{equation}
%
\begin{table}[h!]
\centering
\begin{tabular}{ p{0.09\linewidth}  >{\footnotesize} p{0.5\linewidth}  >{\footnotesize} p{0.27\linewidth} >{\footnotesize} p{0.03\linewidth} }
\hline
$G_{coup}$      		& Heterocellular electrical coupling coefficient		& 0.5 \pers	& ME \\
$P_{IP_{3}}$      		& Heterocellular IP$_{3}$ coupling coefficient	& 0.05 \pers	&  \cite{Koenigsberger2006} \\
$P_{Ca^{2+}}$      		& Heterocellular $P_{Ca^{2+}}$ coupling coefficient	& 0.05 \pers	&  \cite{Koenigsberger2006} \\
\hline
\end{tabular}
\label{tab:JCA3couplingi}
\end{table}
%
\subsubsection*{Additional Equations}
%
Equilibrium distribution of open channel states for the voltage and calcium activated potassium channels (dimensionless):
\begin{equation} \label{eq:Kacti}
K_{act_{i}}= \frac{  \left( \CaConsc + c_{wi}\right)^{2}}    {\left( \CaConsc + c_{wi} \right)^{2}    + \beta_{i} \exp( -\left(   \left[ v_{i}-v_{Ca_{3i}}\right] /R_{Ki}   \right) )      }
\end{equation}
%
Nernst potential of the KIR channel in the SMC (in mV):
\begin{equation}\label{eq:vKIR}
v_{\text{KIR},i} = z_1 [\K]_p-z_2
\end{equation}
%
Conductance of KIR channel (in  \textmu M mV$^{-1}$ s$^{-1}$):
\begin{equation}\label{eq:gKIR}
g_{\text{KIR},i} = \exp(z_5v_i +z_3 [\K]_p - z_4)
\end{equation}
%
%
%
\begin{table}[h!]
\centering
\begin{tabular}{ p{0.09\linewidth}  >{\footnotesize} p{0.5\linewidth}  >{\footnotesize} p{0.27\linewidth} >{\footnotesize} p{0.03\linewidth} }
\hline
$c_{wi}$      			& Translation factor for Ca$^{2+}$ dependence of K$_{Ca}$ channel activation sigmoidal.	& 0.0  \uM	&\cite{Koenigsberger2006} \\
$\beta_{i}$     		& Translation factor for membrane potential dependence of K$_{Ca}$ channel activation sigmoidal.	& 0.13 \uMtwee& \cite{Koenigsberger2006} \\
$v_{Ca_{3i}}$   		& Half-point for the K$_{Ca}$ channel activation sigmoidal.			& -27 \mV	&\cite{Koenigsberger2006} \\
$R_{Ki}$      			& Maximum slope of the K$_{Ca}$ activation sigmoidal.				& 12 \mV	&\cite{Koenigsberger2006} \\
%$z_{1}$      			& Model estimation for membrane voltage KIR channel				& 4.5$\times$10$^3$ \mV	&\cite{Filosa2006}  \\
%$z_{2}$      			& Model estimation for membrane voltage KIR channel			& 112 \mV	&\cite{Filosa2006}  \\
%$z_{3}$      			& Model estimation for the KIR channel conductance				& 4.2$\times$10$^2$ \uMpmVs	&\cite{Filosa2006}  \\
%$z_{4}$      			& Model estimation for the KIR channel conductance				& 12.6 \uMpmVs	&\cite{Filosa2006}  \\
%$z_{5}$      			& Model estimation for the KIR channel conductance			& -7.4$\times$10$^{-2}$ \uMpmVs	&\cite{Filosa2006}  \\
  $ z_1 $	& Model estimation for membrane voltage KIR channel			  & 4.5$\times$10$^3$ \mVpuM & \citep{Filosa2006}\\
  $ z_2 $	& Model estimation for membrane voltage KIR channel			  & 112	 \mV & \citep{Filosa2006}\\
  $ z_3 $	& Model estimation for the KIR channel conductance			  & 4.2$\times$10$^2$ mV$^{-1}$s$^{-1}$ & \citep{Filosa2006}\\
  $ z_4 $	& Model estimation for the KIR channel conductance			  & 12.6			 \uMpmVs & \citep{Filosa2006}\\
  $ z_5 $	& Model estimation for the KIR channel conductance			  & -7.4$\times$10$^{-2}$		 \uM~mV$^{-2}$s$^{-1}$  & \citep{Filosa2006}\\
  \hline
\end{tabular}
\label{tab:Addeq}
\end{table}
\\
%
\subsection{The Contraction Model}
%
Fraction of free phosphorylated cross-bridges (dimensionless):
\begin{equation} \label{eq:dMpdt}
\frac{\dd[Mp]}{\dd t} = K_{4}[AMp] +K_{1} [M] - ( K_{2} + K_{3} ) [Mp]
\end{equation}
%
Fraction of attached phosphorylated cross-bridges (dimensionless):
\begin{equation} \label{eq:dAMpdt}
\frac{\dd[AMp]}{\dd t} =K_{3} [Mp] + K_{6} [AM] - ( K_{4} + K_{5} )[AMp]
\end{equation} 
%
Fraction of attached dephosphorylated cross-bridges (dimensionless):
\begin{equation} \label{eq:dAMdt}
\frac{\dd[AM]}{\dd t} = K_{5} [AMp]-(K_{7}+K_{6})[AM]
\end{equation}
%
Fraction of free non-phosphorylated cross-bridges (dimensionless):
\begin{equation} \label{eq:dMdt}
[M]=1-[AM]-[AMp]-[Mp]
%\frac{\dd[M]}{\dd t} = -K_{1_{i}} [M] + K_{2_{i}} [Mp] + K_{7_{i}} [AM]
\end{equation}
%
Rate constants that represent phosphorylation of M to Mp and of AM to AMp by the active myosin light chain kinase (MLCK), respectively (in \pers):
\begin{equation} \label{eq:gamma}
K_{1} = K_{6} = \gamma_{cross} \CaConsc ^{n_{cross}}
\end{equation}
%
%Note that:
%\begin{equation} \label{eq:fractiesone}
%[AM]+[AMp]+[Mp]+[M]=1
%\end{equation}
%
\begin{table}[h!]
\centering
\begin{tabular}{ p{0.09\linewidth}  >{\footnotesize} p{0.5\linewidth}  >{\footnotesize} p{0.27\linewidth} >{\footnotesize} p{0.03\linewidth} }
\hline
$K_{2}$      	& Rate constant for dephosphorylation (of Mp to M) by myosin light-chain phosphatase (MLCP)																			 & 0.5 \pers & \cite{Hai1989} \\
$K_{3}$      	& Rate constants representing the attachment/detachment of fast cycling phosphorylated crossbridges																	 & 0.4 \pers	& \cite{Hai1989} \\
$K_{4}$      	& Rate constants representing the attachment/detachment of fast cycling phosphorylated crossbridges 																	 & 0.1 \pers	& \cite{Hai1989} \\
$K_{5}$      & Rate constant for dephosphorylation (of AMp to AM) by myosin light-chain phosphatase (MLCP)																			 & 0.5 \pers	& \cite{Hai1989} \\
$K_{7}$      	& Rate constant for latch-bridge detachment					& 0.1 \pers	& \cite{Hai1989} \\
$\gamma_{cross}$      	& Sensitivity of the contractile apparatus to calcium		& 17 \puMdries	& \cite{Koenigsberger2005} \\
$n_{cross}$      		& Fraction constant of the phosphorylation crossbridge				& 3 \Dless	& \cite{Koenigsberger2005} \\
\hline
\end{tabular}
%\caption{This table shows some data}
\label{tab:crossbridge}
\end{table}

\subsection{The Mechanical Model}

Wall thickness of the vessel (in \um):
\begin{equation} \label{eq:h2}
%h=-R+\sqrt{R^2+2R_{0_{pas}}h_{0_{pas}}+h_{0_{pas}}^2}
h=0.1R
\end{equation}
%
Fraction of attached myosin cross-bridges (dimensionless):
\begin{equation}
F_r = [AM_p] + [AM]
\end{equation}
%
Vessel radius (in m):
\begin{equation} \label{eq:dRdt2e}
\dfrac{\mathrm{d}R}{\mathrm{d}t}= \frac{R_{0_{pas}}}{\eta}\left(   \frac{ R P_{T}}{h}  - E(F_r) \frac{R - R_0(F_r)}{R_0(F_r)} \right)
\end{equation}
%
with:
\begin{equation}
E(F_r)= E_{pas} + F_r \left(E_{act} - E_{pas} \right)
\end{equation}
%
\begin{equation}
R_0(F_r)=R_{0_{pas}} + F_r (\alpha_r -1) R_{0_{pas}}
\end{equation}
%
\newpage
\begin{table}[t!]
\centering
\begin{tabular}{ p{0.09\linewidth}  >{\footnotesize} p{0.5\linewidth}  >{\footnotesize} p{0.27\linewidth} >{\footnotesize} p{0.03\linewidth} }
\hline
$\eta   $				& viscosity															& 10$^4$ Pa s 		&  \cite{Koenigsberger2006}\\
$R_{0_{pas}}$			& Radius of the vessel when passive and no stress is applied		& 20  \um 		& ME \\
$P_T$					& Transmural pressure												& 4$\times$10$^3$ \Pa		& ME \\
${E}_{pas}$				& Young's modulus for the passive vessel								& 66$\times$10$^3$ \Pa 		&  \cite{Gore1985}\\
${E}_{act}$				& Young's modulus for the active vessel								& 233$\times$10$^3$ \Pa 	& \cite{Gore1985}\\
$\alpha_r$				& Scaling factor initial radius										& 0.6    		& \cite{Gore1985}\\
\hline
\end{tabular}
%\caption{This table shows some data}
\label{tab:crossbridge}
\end{table}
\par
\par
\par
\par 
$~$
$~$
$~$
$~$
$~$
%\subsection{The pH Model}
%
%The equilibrium distribution of open channel states for the voltage and calcium activated potassium channels:
%\begin{equation} \label{eq:Kacti}
%K_{act_{i}}= \frac{  \left( \CaConsc + c_{w}(pH)\right)^{2}}    {\left( \CaConsc + c_{w}(pH) \right)^{2}    + \beta_{i} exp( -\left(   \left[ v_{i}-v_{Ca_{3i}}\right] /R_{Ki}   \right) )      }
%\end{equation}\\
%%
%The change of $c_w$ over time:
%\begin{equation}\label{eq:pHloop1eq}
%\dfrac{\mathrm{d}c_{w}}{\mathrm{d}t} =  \kappa_i (J_{CICR_i} - P_0 J_{CICR_{ref}})
%\end{equation}\\
%%
%The open probability of the CICR channel as a function of the pH:
%\begin{equation}\label{eq:p0pH}
%P_0=\frac{1}{2} +\frac{1}{2} \tanh \left( \frac{pH - pH_{\frac{1}{2}}}{v_{ramp}}\right)
%\end{equation}
%
%
%%
%\begin{table}[h!]
%\centering
%\begin{tabular}{| p{0.09\linewidth} | >{\footnotesize} p{0.6\linewidth} | >{\footnotesize} p{0.17\linewidth} | >{\footnotesize} p{0.04\linewidth} |}
%\arrayrulecolor{lightgrey}\hline
%$\beta_{i}$     		& Translation factor for membrane potential dependence of K$_{Ca}$ channel activation sigmoidal.	& 0.13 \uMtwee& \cite{Koenigsberger2006} \\
%$v_{Ca_{3i}}$   		& Half-point for the K$_{Ca}$ channel activation sigmoidal.			& -27 \mV	&\cite{Koenigsberger2006} \\
%$R_{Ki}$      			& Maximum slope of the K$_{Ca}$ activation sigmoidal.				& 12 \mV	&\cite{Koenigsberger2006} \\
%$\kappa_{i}$      		& Rate constant of $c_w$											& 0.1 s$^{-1}$	& ME  \\
%$pH_{\frac{1}{2}}$  	& Half-point of $P_0$ sigmodial CICR channel						& 7.175		& ME  \\
%$v_{ramp}$      		& Model estimation for stiffness of $P_0$ sigmodial CICR channel	& 0.15	& ME  \\
%\hline
%\end{tabular}
%\label{tab:Addteq}
%\end{table}