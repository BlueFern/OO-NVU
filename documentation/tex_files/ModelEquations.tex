\documentclass[11pt]{elsarticle}

\usepackage[toc,nonumberlist,nopostdot,acronym,nogroupskip]{glossaries}
\usepackage{xspace} 
\usepackage{longtable}
\usepackage[dvipsnames]{xcolor}
\usepackage{layout}
\usepackage[hidelinks]{hyperref}
\hypersetup{colorlinks = true, citecolor = gray, urlcolor = gray, linkcolor = gray}
\usepackage{amsfonts}
\usepackage[justification=centering]{caption}
\usepackage[numbers]{natbib}
\usepackage{subcaption}
\usepackage{todonotes}
\usepackage{amsmath}
\usepackage{enumitem}
\usepackage[parfill]{parskip}
\usepackage{times}
\usepackage{siunitx}
\sisetup{output-exponent-marker=\ensuremath{\mathrm{e}}}
\usepackage{wrapfig}
\usepackage{adjustbox}
\usepackage{titlesec}
\usepackage{tikz}

\setlength\voffset{-1in}
\setlength\hoffset{-1in}
\setlength\topmargin{2.5cm}
\setlength\oddsidemargin{2.5cm}
\setlength\textheight{23.5cm}
\setlength\textwidth{16.7cm}
\setlength\footskip{1cm}
\setlength\headheight{0cm}
\setlength\headsep{0cm}

\newcommand{\myparagraph}[1]{\paragraph{#1}\mbox{}\\}
\renewcommand*{\glsnamefont}[1]{\textmd{#1}}
%\setglossarystyle{long4col}

\usepackage[parfill]{parskip}

\setcounter{tocdepth}{3}
\setcounter{secnumdepth}{3}

% \Add{} and \Del{} Corrections and \Mark{}
%\usepackage[active,new,noold,marker]{xrcs}
%%%%%%%%%%%%
\newcommand{\sodpot}{Na$^+$/K$^+$\xspace}
\newcommand{\jplc}{J$_{\text{\scriptsize PLC}}$\xspace}
\newcommand{\hco}{HCO$_3^-$\xspace}
\newcommand{\tim}[1]{\textcolor{red}{\textbf{#1}}}
\newcommand{\ums}{$\mu$m s$^{-1}$\xspace}
\newcommand{\mus}{$\mu$M s$^{-1}$\xspace}
\newcommand{\ox}{O$_2$\xspace}
\newcommand{\Cmol}{C mol$^{-1}$\xspace}
\newcommand{\JmolK}{J mol$^{-1}$ K$^{-1}$\xspace}
\newcommand{\perOhmm}{$\Omega^{-1}$\xspace}
\newcommand{\uM}{$\mu$M\xspace}
\newcommand{\um}{$\mu$m\xspace}
\newcommand{\mAcm}{mA cm$^{-2}$\xspace}
\newcommand{\n}{$^{-1}$\xspace}
\newcommand{\e}[1]{\times 10^{#1}}
\newcommand{\psec}{s$^{-1}$\xspace}
\newcommand\ddfrac[2]{\frac{\displaystyle #1}{\displaystyle #2}}

%\newcommand{\na}{Na$^{\text{\scriptsize +}}$\xspace}
%\newcommand{\pot}{K$^{\text{\scriptsize +}}$\xspace}
%\newcommand{\cl}{Cl$^{\text{\scriptsize -}}$\xspace}
%\newcommand{\ca}{Ca$^{\text{\scriptsize 2+}}$\xspace}
%\newcommand{\ip}{IP$_{\text{\scriptsize 3}}$\xspace}
\newcommand{\na}{\gls{na}\xspace}
\newcommand{\pot}{\gls{pot}\xspace}
\newcommand{\cl}{\gls{cl}\xspace}
\newcommand{\ca}{\gls{ca}\xspace}
\newcommand{\ip}{\gls{ip3}\xspace}

\linespread{1.2}

\setcounter{tocdepth}{1}

\newcommand{\td}[1]{\todo[inline, color=GreenYellow, bordercolor=White, size=\normalsize]{\color{Black}#1}}
\newcommand{\vv}{\overline{v_1}} 
\makeatletter
\newcommand*{\rom}[1]{\expandafter\@slowromancap\romannumeral #1@}
\makeatother
\newlength{\drop} % Command for generating a specific amount of whitespace
\drop=0.1\textheight % Define the command as 10% of the total text height

%\newacronym{}{}{}

% \gls{} for normal entry
% \Gls{} for capitalised
% \glspl{} for plural
% \acrfull{} for "<full> (<abbrv>)"
% \acrlong{} for "<full>"

\makenoidxglossaries

\newacronym{smc}{SMC}{smooth muscle cell}
\newacronym{ec}{EC}{endothelial cell}
\newacronym{cicr}{CICR}{Ca$^{2+}$ induced Ca$^{2+}$ release}
\newacronym{nvc}{NVC}{neurovascular coupling}
\newacronym{nvu}{NVU}{neurovascular unit}
\newacronym{sr}{SR}{sarcoplasmic reticulum}
\newacronym{er}{ER}{endoplasmic reticulum}
\newacronym{hopf}{H}{Hopf}
\newacronym{fp}{FP}{fixed point}
%\newacronym{lc}{LC}{limit cycle}
\newacronym{ode}{ODE}{ordinary differential equation}
\newacronym{pde}{PDE}{partial differential equation}
\newacronym{lpc}{LPC}{limit point cycle}
\newacronym{lp}{LP}{limit point}
\newacronym{kir}{KIR}{inward rectifying K$^+$}
\newacronym{pvs}{PVS}{perivascular space}
\newacronym{ne}{NE}{neuron}
\newacronym{sc}{SC}{synaptic cleft}
\newacronym{ac}{AC}{astrocyte}
\newacronym{bt}{BT}{Bogdanov-Takens}
\newacronym{cp}{CP}{Cusp}
\newacronym{gh}{GH}{Generalised Hopf}
\newacronym{ic}{IC}{initial condition}
\newacronym{bc}{BC}{boundary condition}
\newacronym{csd}{CSD}{cortical spreading depression}
\newacronym{mpi}{MPI}{Message Passing Interface}
\newacronym{vtk}{VTK}{Visualisation Toolkit}
\newacronym{ghk}{GHK}{Goldman Hodgkin Katz}
\newacronym{plc}{PLC}{phospholipase-C}
\newacronym{pd}{PD}{Period Doubling}
\newacronym{fhn}{FHN}{FitzHugh-Nagumo}
\newacronym{2d}{2D}{two dimensional}
\newacronym{1d}{1D}{one dimensional}
\newacronym{bk}{BK}{big potassium}
\newacronym{vocc}{VOCC}{voltage operated \gls{ca} channel}
\newacronym{cbf}{CBF}{cerebral blood flow}
\newacronym{ca}{Ca$^{2+}$}{calcium}
\newacronym{pot}{K$^+$}{potassium}
\newacronym{cl}{Cl$^-$}{chlorine}
\newacronym{na}{Na$^+$}{sodium}
\newacronym{ip3}{IP$_3$}{inositol trisphosphate}
\newacronym{atp}{ATP}{adenosine triphosphate}
\newacronym{rhs}{RHS}{right hand side}

\newacronym{trpv}{TRPV4}{transient receptor potential vanniloid-related 4}
\newacronym{no}{NO}{nitric oxide}
\newacronym{20hete}{20-HETE}{20- hydroxyeicosatetraenoic acid}
\newacronym{eet}{EET}{epoxyeicosatrienoic acid}
\newacronym{cox}{COX}{cyclooxegenase enzymes}
\newacronym{aa}{AA}{arachidonic acid}
\newacronym{PgE2}{PgE$_2$}{prostaglandin E$_2$}
\newacronym{ecs}{ECS}{extracellular space}
\newacronym{wss}{WSS}{wall sheer stress}
\newacronym{nmda}{NMDA}{N-methyl-D-aspartate}
\newacronym{sgc}{sGC}{soluble guanylyl cyclase}
\newacronym{cgmp}{cGMP}{cyclic guanosine monophosphate}
\newacronym{mglur}{mGluR}{metabotropic glutamate receptor}

\newacronym{efs}{EFS}{electro field stimulation}
\newacronym{mlc}{MLC}{myosin light chain kinase}
\newacronym{pkc}{PKC}{protein-kinase C}
\newacronym{cpi17}{CPI-17}{myosin phosphatase inhibitor protein}

\newacronym{bold}{BOLD}{blood-oxygen-level dependent}
\newacronym{fmri}{fMRI}{functional magnetic resonance imaging}
\newacronym{cbv}{CBV}{cerebral blood volume}
\newacronym{nat}{NaT}{transient Na$^+$}

\newacronym{sodpot}{Na$^+$/K$^+$}{sodium potassium}
\newacronym{nnos}{nNOS}{neuronal \gls{no} synthase}
\newacronym{enos}{eNOS}{endothelial \gls{no} synthase}
\newacronym{cmro2}{CMRO$_2$}{cerebral metabolic rate of oxygen}
\newacronym{hbo}{HbO}{oxyhemoglobin}
\newacronym{hbr}{HbR}{deoxyhemoglobin}
\newacronym{hbt}{HbT}{total hemoglobin}
\newacronym{mtt}{MTT}{mean transit time}
\newacronym{ltp}{LTP}{long term potentiation}
\newacronym{epsp}{EPSP}{excitatory postsynaptic potential}
\newacronym{ipsp}{IPSP}{inhibitory postsynaptic potential}
\newacronym{lc}{LC}{locus coeruleus}
\newacronym{lcna}{LC-NA}{noradrenalin locus coeruleus}
\newacronym{sac}{SAC}{stretch activated channel}

\begin{document}


\begin{frontmatter}
	
	\title{\vspace{-2cm}Supplementary material: Model equations and parameters}
	
	
	\author[1]{Allanah Kenny\corref{3}}
	\cortext[3]{Corresponding author, \textit{allanah.kenny@pg.canterbury.ac.nz} }
	\author[2]{Michael J. Plank}
	\author[1]{Tim David}
	
	\address[1]{High Performance Computing Centre, University of Canterbury, New Zealand}
	\address[2]{Department of Mathematics and Statistics, University of Canterbury, New Zealand}
	
\end{frontmatter}

The single \gls{nvu} model originally developed by \citet{Farr2011} and later extended by \citet{Dormanns2015}, \citet{Dormanns2016}, \citet{Mathias2017}, \citet{Kenny2018}, and (Elshin ref) contains 67 \glspl{ode} plus a large number of algebraic variables and parameters. The equations and parameters are divided into sections corresponding to different compartments and pathways of the model.
The following parameters are given for ordinary \gls{nvc} conditions.

\section{Global Constants}
\begin{longtable}[h!]{ p{0.12\linewidth}   p{0.6\linewidth}   p{0.28\linewidth} }
	\hline
	Parameter & Description & Value \\
	\hline

$F$			& Faraday's constant 	& 96.485 C mmol\n				\\
$R_g$ 			& Gas constant 				& 8.315 \JmolK	 \\
$T$ 	    	& Temperature constant								& 300 K	\\
$\phi$ 		& $R_g T / F$				& 26.7 mV \\		
$z_K$			& Ionic valence for \pot 										& 1   \\ 
$z_{Na}$			& Ionic valence for \na 											& 1   \\ 
$z_{Cl}$			& Ionic valence for \cl 											& $-1$  \\ 
$z_{NBC}$ 		& Effective valence of the NBC cotransporter complex 				& $-1$ \\
$z_{Ca}$  & Ionic valence for \ca & 2 \\
	\hline
\end{longtable}


\section{Neuron and Extracellular Space}

\subsection{Input to the model}
The input current to the soma (mA cm$^{-2}$):
	\begin{equation}
%	I_{stim} = I_{strength} \cdot \text{rectpuls}(t - t_0, lengthpulse)
	I_{stim} = 
	\begin{cases} 
	      0 & t\leq t_0 \\
	      I_{strength} & t_0\leq t\leq t_f \\
	      0 & t_f\leq t 
	\end{cases}
	\end{equation}
%
\subsection{\Glspl{ode}}
The membrane potential of the soma/axon ($v_{sa}$) and dendrite ($v_d$) (mV):
	\begin{align}
	C_{m}\frac{d v_{sa}}{d t} 	&= -I_{tot_{sa}}+\frac{1}{2R_a\delta_d^{2}}(v_d-v_{sa})+I_{stim} \\
	C_{m}\frac{d v_d}{d t} 		&= -I_{tot_d}+\frac{1}{2R_a\delta_d^{2}}(v_{sa}-v_d)
	\end{align}
%	
The ion concentrations (\pot and \na) in the soma/axon (mM):
	\begin{align} 
	\frac{d K_{sa}}{dt} &= -\frac{A_s}{F V_s} I_{K,tot_{sa}} + \frac{D_{K,n}(V_d + V_s)} {2 \delta_d^2 V_s} (K_d - K_{sa}) \\
	\frac{d Na_{sa}}{dt} &= -\frac{A_s}{F V_s} I_{Na,tot_{sa}} + \frac{D_{Na,n}(V_d + V_s)} {2 \delta_d^2 V_s} (Na_d - Na_{sa})
	\end{align}
%
The ion concentrations (\pot and \na) in the dendrite (mM):
	\begin{align} 
	\frac{d K_{d}}{dt} &= -\frac{A_d}{F V_d} I_{K,tot_{d}} + \frac{D_{K,n}(V_d + V_s)} {2 \delta_d^2 V_d} (K_d - K_{d}) \\
	\frac{d Na_{d}}{dt} &= -\frac{A_d}{F V_d} I_{Na,tot_{d}} + \frac{D_{Na, n}(V_d + V_s)} {2 \delta_d^2 V_d} (Na_d - Na_{d})
	\end{align}
%
The \pot and \na ion concentrations in the \gls{ecs} (mM):	
	\begin{align}
	\frac{d Na_e}{dt} &= \frac{1}{F f_e} \left( \frac{A_s I_{Na,tot_{sa}}}{V_s} + \frac{A_d I_{Na,tot_d}}{V_d} \right) \\
	\frac{d K_e}{dt} &= \frac{1}{F f_e} \left( \frac{A_s I_{K,tot_{sa}}}{V_s} + \frac{A_d I_{K,tot_d}}{V_d} \right) - \frac{d \text{Buff}_e}{dt}
	\end{align}	
%
The buffer concentration in the \gls{ecs} ($\text{Buff}_e$) (mM):
	\begin{align}
	\frac{d\text{Buff}_e}{dt} &= \mu (B_0 - \text{Buff}_e) \frac{K_e}{1 + \exp\left( \frac{-(K_e - 5.5)}{1.09} \right)} - \mu \text{Buff}_e
	\end{align}
%
Activation gating variable $m_*$ (for $m_1$ to $m_8$) (-):
	\begin{align}
	\frac{d m_*}{dt} = \left( \alpha_{m_*} (1-m_*) - \beta_{m_*} m_*  \right)
	\end{align}
%	
Deactivation gating variable $h_*$ (for $h_1$ to $h_6$) (-):
	\begin{align}
	\frac{d h_*}{dt} =  \left( \alpha_{h_*} (1-h_*) - \beta_{h_*} h_*  \right)
	\end{align}
	%
The tissue oxygen concentration (mM):	
	\begin{align}
	\frac{d O_2}{dt} = J_{O_2 \: vascular} - J_{O_2 \: background} - J_{O_2 \: pump}
	\end{align}
%
\subsection{Algebraic Variables}
%
Total current of ions in soma/axon or dendrite (*) (\mAcm):
\begin{align}
I_{tot_{*}} =I_{K,tot_{*}} +  I_{Na,tot_{*}} + I_{leak_{*}}
\end{align}
%
Total current of \pot in soma/axon (\mAcm):
\begin{align}
I_{K,tot_{sa}} = I_{KDR_{sa}} + I_{KA_{sa}} + I_{K,leak_{sa}} + I_{pump,K_{sa}}
\end{align}
%
Total current of \pot in dendrite (\mAcm):
\begin{align}
I_{K,tot_{d}} = I_{KDR_{d}} + I_{KA_{d}} + I_{K,leak_{d}} + I_{pump,K_{d}} + I_{NMDA,K_d}
\end{align}
%
Total current of \na in soma/axon (\mAcm):
\begin{align}
I_{Na,tot_{sa}} = I_{NaP_{sa}} + I_{NaT_{sa}} + I_{Na,leak_{sa}} + I_{pump,Na_{sa}}
\end{align}
%
Total current of \na in dendrite (\mAcm):
\begin{align}
I_{Na,tot_{d}} = I_{NaP_{d}} + I_{Na,leak_{d}} + I_{pump,Na_{d}} + I_{NMDA,Na_d}
\end{align}
%
\pot current through \sodpot \gls{atp}-ase pump in the soma/axon or dendrite (*) (\mAcm):
\begin{align}
I_{pump,K_{*}} &= -2 \: I_{pump_{*}}
\end{align}
%
\na current through \sodpot \gls{atp}-ase pump in soma/axon or dendrite (*) (\mAcm):
\begin{align}
I_{pump,Na_{*}} &= 3 \: I_{pump_{*}}
\end{align}
%
 \sodpot \gls{atp}-ase pump flux in soma/axon or dendrite (*) (\mAcm):
\begin{align}
I_{pump_{*}} &= I_{max} J_{pump1_{*}} J_{pump2}(O_2)
\end{align}
%
Oxygen independent term of the \gls{atp}-ase pump in both the soma/axon or dendrite (*) (-):
	\begin{align}
	J_{pump1_{*}} &= \left( 1 + \frac{ K_{e,0} }{K_e} \right)^{-2} \left( 1 + \frac{ Na_{*,0} }{Na_{*}} \right)^{-3} 
	\end{align}
%
Oxygen dependent term of the \gls{atp}-ase pump (-): 
	\begin{align}
	J_{pump2}(O_2) = 2 \left( 1 + \frac{ O2_0 }{ (1 - \alpha_{O2}) \: O_2 + \alpha_{O2} \: O2_0  }    \right)^{-1}
	\end{align}
%
The vascular supply of oxygen (mM \psec):	
	\begin{align}
	J_{O_2 \: vascular} = J \frac{O2_b - O_2}{O2_b - O2_0}
	\end{align}
%	
The background oxygen consumption (mM \psec):
	\begin{align}
	J_{O_2 \: background} = J_0 P_{O2} (1 - \gamma_{O2})
	\end{align}
%	
The tissue oxygen consumption due to the \gls{atp}-ase pump (mM \psec):
	\begin{align}
	J_{O_2 \: pump} = J_0 P_{O2} \gamma_{O2} \frac{J_{pump1_{sa}} + J_{pump1_{d}}}{J_{pump1_{sa_0}} + J_{pump1_{d_0}}}
	\end{align}
%
The change in oxygen concentration due to \gls{cbf} (mM \psec):
	\begin{align}
	J = J_0 \frac{\text{CBF}}{\text{CBF}_{init}}
	\end{align}
%
The cerebral blood flow (-):
	\begin{align}
	\text{CBF} =  \text{CBF}_{init} \frac{R^4}{R_{init}^4}
	\end{align}
%
The normalised pump rate (-):
	\begin{align}
	P_{O2} = \frac{J_{pump2}(O_2) - J_{pump2}(0)}{J_{pump2}(O2_0) - J_{pump2}(0)}
	\end{align}
%
Leak currents of \pot, \na and general leak in the soma/axon or dendrite (\mAcm):
	\begin{align}
	I_{K,leak_{*}} &= g_{K,leak_{*}} (v_{*} - E_{K_{*}}) \\
	I_{Na,leak_{*}} &= g_{Na,leak_{*}} (v_{*} - E_{Na_{*}}) \\
	I_{leak_{*}} &= g_{leak_{*}} (v_{*} - E_{leak_{*}}) 
	\end{align}
%
Nernst potential for \pot and \na in the soma/axon or dendrite (*), (mV):
\begin{align}
E_{K_{*}} &= \frac{\phi}{z_K} \ln \left(\frac{K_e}{K_{*}}\right) \\
E_{Na_{*}} &= \frac{\phi}{z_{Na}} \ln \left(\frac{Na_e}{Na_{*}}\right)
\end{align}
%		
\pot current through KDR channel in the soma/axon (\mAcm):
\begin{align}
I_{KDR_{sa}} = m_2^2 \; \frac{  g_{KDR} F v_{sa} \left( K_{sa} - \exp \left( \frac{-v_{sa}}{\phi}  \right) K_e  \right)   }{   \phi \left( 1 - \exp  \left(  \frac{-v_{sa}}{\phi}  \right)  \right)   }
\end{align}
%
\pot current through KA channel in the soma/axon (\mAcm):
\begin{align}
I_{KA_{sa}} = m_3^2 \; h_2 \; \frac{  g_{KA} F v_{sa} \left( K_{sa} - \exp \left( \frac{-v_{sa}}{\phi}  \right) K_e  \right)   }{   \phi \left( 1 - \exp  \left(  \frac{-v_{sa}}{\phi}  \right)  \right)   }
\end{align}
%
\pot current through KDR channel in the dendrite (\mAcm):
\begin{align}
I_{KDR_{d}} = m_6^2 \; \frac{  g_{KDR} F v_{d} \left( K_{d} - \exp \left( \frac{-v_{d}}{\phi}  \right) K_e  \right)   }{   \phi \left( 1 - \exp  \left(  \frac{-v_{d}}{\phi}  \right)  \right)   }
\end{align}
%		
\pot current through KA channel in the dendrite (\mAcm):
\begin{align}
I_{KA_{d}} = m_7^2 \; h_5 \; \frac{  g_{KA} F v_{d} \left( K_{d} - \exp \left( \frac{-v_{d}}{\phi}  \right) K_e  \right)   }{   \phi \left( 1 - \exp  \left(  \frac{-v_{d}}{\phi}  \right)  \right)   }
\end{align}
%
\pot current through \gls{nmda} channel in the dendrite (\mAcm):
\begin{align}
I_{NMDA,K_{d}} = m_5 \; h_4 \; \frac{  g_{NMDA} F v_{d} \left( K_{d} - \exp \left( \frac{-v_{d}}{\phi}  \right) K_e  \right)   }{   \phi \left( 1 - \exp  \left(  \frac{-v_{d}}{\phi}  \right)  \right)  \Big(  1 + 0.33 \: [Mg]_0 \exp (-0.07 v_d - 0.7 ) \Big) } 
\end{align}		
%
\na current through NaP channel in soma/axon (\mAcm):
\begin{align}
I_{NaP,Na_{sa}}    = m_1^2 \; h_1 \; \frac{  g_{NaP} F v_{sa} \left( Na_{sa} - \exp \left( \frac{-v_{sa}}{\phi}  \right) Na_e  \right)   }{   \phi \left( 1 - \exp  \left(  \frac{-v_{sa}}{\phi}  \right)  \right)   }
\end{align}
%
\na current through NaT channel in soma/axon (\mAcm):
\begin{align}
I_{NaT,Na_{sa}}    = m_8^3 \; h_6 \; \frac{  g_{NaT} F v_{sa} \left( Na_{sa} - \exp \left( \frac{-v_{sa}}{\phi}  \right) Na_e  \right)   }{   \phi \left( 1 - \exp  \left(  \frac{-v_{sa}}{\phi}  \right)  \right)   }
\end{align}
%		
\na current through NaP channel in dendrite (\mAcm):
\begin{align}
I_{NaP,Na_{d}}    = m_4^2 \; h_3 \; \frac{  g_{NaP} F v_{d} \left( Na_{d} - \exp \left( \frac{-v_{d}}{\phi}  \right) Na_e  \right)   }{   \phi \left( 1 - \exp  \left(  \frac{-v_{d}}{\phi}  \right)  \right)   }
\end{align}
%
\na current through \gls{nmda} channel in dendrite (\mAcm):
\begin{align}
I_{NMDA,Na_{d}} = m_5 \; h_4 \; \frac{  g_{NMDA} F v_{d} \left( Na_{d} - \exp \left( \frac{-v_{d}}{\phi}  \right) Na_e  \right)   }{   \phi \left( 1 - \exp  \left(  \frac{-v_{d}}{\phi}  \right)  \right)  \Big(  1 + 0.33 \: [Mg]_0 \exp (-0.07 v_d - 0.7 ) \Big) } 
\end{align}		
%
Rate functions for the activation $m$ and deactivation $h$ gating variables (\psec):
	\begin{align}
	&\alpha_{m_1} = \frac{1000}{6}\frac{1}{ 1 + \exp(- (0.143 v_{sa} + 5.67))  } \\
	&\beta_{m_1} = \frac{1000}{6} - \alpha_{m_1}
	\end{align}
	%
	\begin{align}
	&\alpha_{m_2} = 16 \frac{v_{sa} + 34.9}{1 - \exp(-(0.2 v_{sa} + 6.98))} \\
	&\beta_{m_2} = 250 \exp(-(0.025 v_{sa} + 1.25))
	\end{align}
	%
	\begin{align}
	&\alpha_{m_3} = 20 \frac{v_{sa} + 56.9}{1 - \exp(-(0.1 v_{sa} + 5.69))} \\
	&\beta_{m_3} = 17.5 \frac{v_{sa} + 29.9}{\exp(0.1 v_{sa} + 2.99) - 1}
	\end{align}
	%
	\begin{align}
	&\alpha_{m_4} = \frac{1000}{6} \frac{1}{ 1 + \exp(- (0.143 v_{d} + 5.67)) } \\
	&\beta_{m_4} = \frac{1000}{6} - \alpha_{m_4}
	\end{align}
	%
	\begin{align}
	&\alpha_{m_5} = 500 \frac{1}{1 + \exp(\frac{13.5 - K_e}{1.42})} \\
	&\beta_{m_5} = 500 - \alpha_{m_5}
	\end{align}
	%
	\begin{align}
	&\alpha_{m_6} = 16 \frac{v_{d} + 34.9}{1 - \exp(-(0.2 v_{d} + 6.98))} \\
	&\beta_{m_6} = 250 \exp(-(0.025 v_{d} + 1.25))
	\end{align}
	%
	\begin{align}
	&\alpha_{m_7} = 20 \frac{v_{d} + 56.9}{1 - \exp(-(0.1 v_{d} + 5.69))} \\
	&\beta_{m_7} = 17.5 \frac{v_{d} + 29.9}{\exp(0.1 v_{d} + 2.99) - 1}
	\end{align}
	%
	\begin{align}
	&\alpha_{m_8} = 320 \frac{-v_{sa} - 51.9}{\exp(-(0.25 v_{sa} + 12.975)) - 1} \\
	&\beta_{m_8} = 280 \frac{v_{sa} + 24.89}{\exp(0.2 v_{sa} + 4.978) - 1} 
	\end{align}
	%
	\begin{align}
	&\alpha_{h_1} = 5.12 \times 10^{-5} \exp(-(0.056 v_{sa} + 2.94)) \\
	&\beta_{h_1} = 1.6 \times 10^{-3} \frac{1}{1 + \exp(-(0.2 v_{sa} + 8))}
	\end{align}
	%
	\begin{align}
	&\alpha_{h_2} = 16 \exp(-(0.056 v_{sa} + 4.61)) \\
	&\beta_{h_2} = 500 \frac{1}{1 + \exp(-(0.2 v_{sa} + 11.98))}
	\end{align}
	%
	\begin{align}
	&\alpha_{h_3} = 5.12 \times 10^{-5} \exp(-(0.056 v_{d} + 2.94)) \\
	&\beta_{h_3} = 1.6 \times 10^{-3} \frac{1}{1 + \exp(-(0.2 v_{d} + 8))}
	\end{align}
	%
	\begin{align}
	&\alpha_{h_4} = 0.5 \frac{1}{\left( 1 + \exp \left( \frac{K_e - 6.75}{0.71} \right)  \right)} \\
	&\beta_{h_4} =0.5 - \alpha_{h_4}
	\end{align}
	%
	\begin{align}
	&\alpha_{h_5} = 16 \exp(-(0.056 v_{d} + 4.61)) \\
	&\beta_{h_5} = 500 \frac{1}{1 + \exp(-(0.2 v_{d} + 11.98))} 
	\end{align}
	%
	\begin{align}
	&\alpha_{h_6} = 128 \exp(-(0.056 v_{sa} + 2.94)) \\
	&\beta_{h_6} = 4\e{3} \frac{1}{1 + \exp(-(0.2 v_{sa} + 6))}
	\end{align}
%	
\begin{longtable}[h!]{ p{0.12\linewidth}   p{0.6\linewidth}   p{0.28\linewidth} }
	\hline
	Parameter & Description & Value \\
	\hline
$I_{strength}$ 		& Amplitude of input current 		& 0.022 \mAcm 		  \\
$t_0$				& Start time of input current			& 0 s	  \\
$t_f$		& Final time of input current 			& 20 s		  \\
$O2_b$ 		& Blood oxygen level							& 0.04 mM		\\
$O2_0$ 		& Equilibrium tissue oxygen level				& 0.02 mM 		\\
$\gamma_{O2}$ 		& Fraction of the total oxygen consumption at steady state		& 0.1 		 \\
$J_0$		& Equilibrium change in oxygen concentration due to \gls{cbf}						& 0.032 mM \psec		\\
$\text{CBF}_{init}$		& Equilibrium \gls{cbf}						& 0.032 		\\
$J_{pump1_{sa_0}}$ 		& Steady state pump rate in the soma/axon			& 0.0312 		 \\
$J_{pump1_{d_0}}$ 		& Steady state pump rate in the dendrite		& 0.0312 		 \\
$R_{init}$ 		& Vessel radius when passive and no stress is applied				& 20 \um 		 \\
$\alpha_{O2}$ 		& Fraction of oxygen independent \gls{atp} production		& 0.05 		 \\
$J_{pump2}(0)$ 		& Pump rate when oxygen concentration is 0		& 0.0952 		\\
$J_{pump2}(O2_0)$ 		& Pump rate when oxygen is at equilibrium			& 1 	 \\
$K_{e,0}$ 		& Equilibrium $K_e$		& 2.9 mM	 \\
$Na_{sa,0}$ 		& Equilibrium $Na_{sa}$											& 10 mM		\\
$Na_{d,0}$ 		& Equilibrium $Na_d$										& 10 mM	 \\
$C_m$ 		& Membrane capacitance 		& 7.5$\times 10^{-7}$ S cm$^{-2}$ s 	 \\
$R_a$		& Input resistance of dendritic tree & 1.83$\times 10^5$ $\Omega$ \\ 
$\delta_d$	& Half length of dendrite	& 4.5$\times 10^{-2}$ cm \\
$A_s$ 		& Soma/axon surface area		& 1.586$\times 10^{-5}$ cm$^2$ 	 \\
$A_d$ 		& Dendrite surface area	& 2.6732$\times 10^{-4}$ cm$^2$  \\
$V_s$ 		& Soma/axon volume			& 2.16$\times 10^{-9}$ cm$^3$ 		\\
$V_d$ 		& Dendrite volume		& 5.614$\times 10^{-9}$ cm$^3$ 		 \\
$f_e$		& \gls{ecs} to neuron volume ratio	& 0.15								 \\
$D_{Na, n}$	& Intracellular diffusion rate of \na 		& 1.33$\times 10^{-5}$ cm$^2$ \psec		\\
$D_{K,n}$		& Intracellular diffusion rate of \pot 		& 1.96$\times 10^{-5}$ cm$^2$ \psec		 \\
$g_{K,leak_{sa}}$ & Conductance of \pot leak channel on soma/axon	&  $2.1989\e{-4}$ S cm$^{-2}$ \\		
$g_{Na,leak_{sa}}$ & Conductance of \na leak channel on soma/axon	& $6.2378\e{-5}$  S cm$^{-2}$\\	
$g_{leak_{sa}}$ 	& Conductance of general leak channel on soma/axon		&  $6.2378\e{-4}$ S cm$^{-2}$  \\	
$g_{K,leak_{d}}$ 	& Conductance of \pot leak channel on dendrite	&  $2.1987\e{-4}$  S cm$^{-2}$ \\	
$g_{Na,leak_{d}}$ 	& Conductance of \na leak channel on dendrite	&  $6.2961\e{-5}$  S cm$^{-2}$ \\	
$g_{leak_{d}}$ 		& Conductance of general leak channel on dendrite	&  $6.2961\e{-4}$ S cm$^{-2}$  \\			
$E_{leak_{sa}}$			& Nernst potential for general leak in the soma/axon			& $-70$ mV				 \\
$E_{leak_{d}}$			& Nernst potential for general leak in the dendrite			& $-70$ mV	 \\				
$g_{KDR}$ 		& KDR channel conductance		& $10^{-4}$  S cm$^{-2}$  \\		
$g_{KA}$ 		& KA channel conductance		& $10^{-5}$  S cm$^{-2}$  \\		
$g_{NMDA}$ 		& \gls{nmda} channel conductance		& $10^{-5}$ S cm$^{-2}$  \\		
$g_{NaP}$ 		& NaP channel conductance		& $2\e{-6}$   S cm$^{-2}$ \\					
$g_{NaT}$ 		& NaT channel conductance		& $10^{-5}$  S cm$^{-2}$ \\
$[Mg]_0$ 		& Equilibrium magnesium 		& 1.2 mM L\n  \\		
$I_{max}$ 		& Maximum rate of \sodpot \gls{atp}-ase pump		& $0.078$ \mAcm  \\		
$\mu$ 		& Buffer rate											& $8\times10^{-4}$ m\psec 		 \\
$B_0$		& Effective total buffer concentration 		& 500 mM \\
				\hline
\caption{Parameters of the neuron and extracellular space submodel, for references see \citet{Mathias2017}.}
\end{longtable}
%
\section{\Gls{bold} response}

\subsection{\Glspl{ode}}
The non dimensional \gls{cbv} (-):
\begin{align}
\frac{d \text{CBV}}{dt} = \frac{1}{\tau_{MTT} + \tau_{TAT}} \left( \frac{\text{CBF}}{\text{CBF}_{init}} - \text{CBV}^{d} \right)
\end{align}
%	
The non dimensional \gls{hbr} concentration (-):
\begin{align}
\frac{d \text{HbR}}{dt} = \frac{1}{\tau_{MTT}} \left(  \frac{\text{CMRO}_2}{\text{CMRO}_{2_{0}}}
- \frac{\text{HbR}}{\text{CBV}} f_{out} 
\right)
\end{align}
%
\subsection{Algebraic Variables}	
The non dimensional normalised \gls{hbt} concentration (-):	
\begin{align}
\text{HbT}_N = \frac{ \text{CBF}_N \text{HbR}_N }{\text{CMRO}_{2_N} }
\end{align}
%
where the normalised \gls{cbf} is given by CBF$_N = $ CBF/CBF$(0)$ and CBF$(0)$ is the steady state value, similarly for \gls{hbr} and $\text{CMRO}_{2_N}$.

The non dimensional normalised \gls{hbo} concentration (-):
\begin{align}
\text{HbO}_N = \text{HbT}_N - \text{HbR}_N + 1
\end{align}
%
The time dependent outflow from the venous compartment (-):
\begin{align}
f_{out} = \text{CBV}^{d} + \tau_{TAT} \frac{d \text{CBV}}{dt}
\end{align}
%	
The cerebral metabolic rate of oxygen consumption (mM \psec):
\begin{align}
\text{CMRO}_2 = J_{O_2 \: background} + J_{O_2 \: pump}
\end{align}
%	
The equilibrium value of $\text{CMRO}_2$ (mM \psec):
\begin{align}
\text{CMRO}_{2_0} =  J_0 P_{02}
\end{align}
%	
The oxygen extraction fraction (-):
\begin{align}
E = E_0\frac{\text{CMRO}_2}{J}
\end{align}
%
The \gls{bold} signal change from its steady state value (-):
\begin{align}
\Delta BOLD \approx V_0 \left( a_1 \left[ 1 - \text{HbR}_N \right] + a_2 \left[ \text{CBV}_N - 1 \right] \right) 
\label{eq:bold}
\end{align}
\xspace
%	
\begin{longtable}[h!]{ p{0.12\linewidth}   p{0.6\linewidth}   p{0.28\linewidth} }
	\hline
	Parameter & Description & Value \\
	\hline
	$\tau_{MTT}$ 		& Mean transit time											&  3 s		\\
	$\tau_{TAT}$ 				& Transient adjustment time constant						& 20 s		\\
	$ \text{CBF}_{init}$				& Steady state \gls{cbf}							& 0.032  	\\
	$d$					& Empirical relation between \gls{cbf} and \gls{cbv}					& 2.5		\\	
	$a_1$ 				& Weight for \gls{hbr} change											& 3.4		\\
	$a_2$ 				& Weight for \gls{cbv} change											& 1 		\\
	$V_0$				& Resting venous blood volume fraction 						& 0.03		\\
	$E_0$				& Baseline oxygen extraction fraction						& 0.4		\\
\hline
\caption{Parameters of the \gls{bold} submodel, for references see \citet{Mathias2017}.}
\end{longtable}

\section{Synaptic Cleft and Astrocyte}
%
\subsection{\Glspl{ode}}
%
\pot concentration in the \gls{sc} (\uM):
\begin{align}
\frac{d K_s}{dt} = \frac{1}{VR_{sk}} \left( J_{K_k} - 2 J_{NaK_k} - J_{NKCC1_k} - J_{KCC1_k} \right) + J_{NEtoSC}
\end{align}
%
\na concentration in the \gls{sc} (\uM):
\begin{align}
\frac{d Na_s}{dt} =  \frac{1}{VR_{sk}} \left( J_{Na_k} + 3*J_{NaK_k} - J_{NKCC1_k} - J_{NBC_k} \right) - J_{NEtoSC}
\end{align}
%
\hco concentration in the \gls{sc} (\uM):
\begin{equation} \label{eq:HCOEx}
\frac{d HCO_{3_{s}}}{dt}=  \frac{1}{VR_{sk}} \left( -2 J_{NBC_{k}} \right)
\end{equation} 
%
\pot concentration in the astrocyte (\uM):
\begin{equation} \label{eq:KInt}
\frac{d K_k}{dt}=- J_{K_k} + 2 J_{NaK_{k}} + J_{NKCC1_{k}} +  J_{KCC1_{k}}
- J_{BK_k}  
\end{equation}
%
\na concentration in the astrocyte (\uM):
\begin{equation} \label{eq:NaInt}
\frac{d Na_k}{dt}=-J_{Na_k} - 3 J_{NaK_{k}} + J_{NKCC1_{k}} +  J_{NBC_{k}}
\end{equation}
%
\hco concentration in the astrocyte (\uM):
\begin{equation} \label{eq:HCOInt}
\frac{d HCO_{3_k}}{dt}= 2 J_{NBC_{k}} 
\end{equation}
%
\cl concentration in the astrocyte (\uM):
\begin{equation} \label{eq:ClInt}
\frac{d Cl_k}{dt}= \frac{dNa_k}{dt} + \frac{dK_k}{dt} - \frac{d HCO_{3_{k}}}{dt} + 2 \frac{d Ca_k}{dt}
\end{equation}
%
The astrocytic cytosolic \ca concentration (\uM):
\begin{equation}
\frac{d Ca_k}{dt} = B_{cyt} \left( J_{IP3_k} - J_{pump_k} + J_{ERleak_k} - \frac{J_{TRPV_k}}{r_{buff}} \right)
\label{eq:ACca}
\end{equation}	
%
The astrocytic \ip concentration (\uM):
\begin{equation}
\frac{d IP3_k}{dt} = r_h G - k_{deg} \; IP3_k
\end{equation}
%
The astrocytic \gls{eet} concentration (\uM):
\begin{equation}
\frac{d eet_k}{dt} = V_{eet} \max(Ca_k - c_{k_{min}}, 0) - k_{eet} eet_k
\end{equation}
%
The \ca concentration in the astrocytic \gls{er} (\uM):
\begin{equation}
\frac{d s_k}{dt} =  \frac{ -B_{cyt} }{VR_{ERcyt}} \left( J_{IP3_k} - J_{pump_k} + J_{ERleak_k} \right)
\end{equation}
%
Membrane potential of the \gls{ac} (mV):
\begin{equation} \label{eq:v_k}
\frac{d v_k}{dt} = \gamma_j ( -J_{BK_k} - J_{K_k} - J_{Cl_k} - J_{NBC_k} - J_{Na_k} - J_{NaK_k} - 2J_{TRPV_k})
\end{equation}
%
The open probability of the \gls{bk} channel (-):
\begin{equation}
\frac{d w_k}{dt} = \phi_n ( w_{\infty} - w_k)
\end{equation} 
%
The inactivation variable $h_k$ of the astrocytic IP$_3$R channel (-):
\begin{equation}
\frac{d h_k}{dt} = k_{on} \left[ K_{inh} - (Ca_k + K_{inh}) h_k \right]
\end{equation}


			\subsection{Algebraic Variables}
The glutamate concentration in the \gls{sc} (\uM):
\begin{align}
Glu = \frac{Glu_{max}}{2} \left( 1 + \tanh \left( \frac{K_e - K_{e_{switch}}}{Glu_{slope}}  \right)  \right)
\end{align}
%				
The flux of \pot into the \gls{sc} based on the extracellular \pot ($\mu$Ms$^{-1}$):
\begin{align}
J_{NEtoSC} = c_{unit} k_{syn}  \frac{d K_e}{dt}   
\end{align}
%
\cl concentration in the \gls{sc} (\uM): 
\begin{equation} \label{eq:ClEx}
Cl_s = Na_s + K_s - HCO_{3_s}
\end{equation}
%
\cl flux through the \cl channel (\mus): 
\begin{equation} \label{eq:J_Cl}
J_{Cl_k}=G_{Cl_{k}}(v_k - E_{Cl_k})
\end{equation}
%
\pot flux through the \pot channel (\mus): 
\begin{equation} \label{eq:J_K}
J_{K_k}=G_{K_{k}}(v_k - E_{K_k})
\end{equation}
%
\na flux through the \na channel (\mus):
\begin{equation} \label{eq:J_Na}
J_{Na_k}=G_{Na_{k}}(v_k - E_{Na_k})
\end{equation}
%
\na and \hco flux through the NBC channel (\mus): 
\begin{equation} \label{eq:J_NBC}
J_{NBC_k}=G_{NBC_k}\left(  v_k -E_{NBC_k}  \right)
\end{equation}
%
\cl and \pot flux through the KCC1 channel (\mus): 
\begin{equation} \label{eq:J_KCC1}
J_{KCC1_k}=G_{KCC1_k} \phi \ln \left(\frac{K_s Cl_s }{K_k Cl_k}\right)
\end{equation}
%
\na, \pot and \cl flux through the NKCC1 channel (\mus): 
\begin{equation} \label{eq:J_NKCC1}
J_{NKCC1_k}=G_{NKCC1_k} \phi \ln \left(\frac{Na_s K_s {Cl_s}^2}{Na_k K_k {Cl_k}^2}\right)
\end{equation}
%
Flux through the \sodpot \gls{atp}-ase pump (\mus): 
\begin{equation} \label{eq:J_NaK_s}
J_{NaK_{k}}=J_{NaK_{max}}\frac{{Na_k}^{1.5}}{{Na_k}^{1.5}+{K_{Na_k}}^{1.5}}\frac{K_s}{K_s+K_{K_s}}
\end{equation}
%
\pot flux through the \gls{bk} channel (\mus): 
\begin{equation} \label{eq:J_BK}
J_{BK_k}= G_{BK_k}w_k\left( v_k-E_{BK_k} \right)
\end{equation}
%
Nernst potential for the \pot channel (mV):
\begin{equation} \label{eq:E_K}
E_{K_k}=\frac{\phi}{z_K}\ln\left( \frac{K_s}{K_k}\right) 
\end{equation}
%
Nernst potential for the \na channel (mV):
\begin{equation} \label{eq:E_Na}
E_{Na_k}=\frac{\phi}{z_{Na}}\ln\left( \frac{Na_s}{Na_k}\right) 
\end{equation}
%
Nernst potential for the \cl channel (mV):
\begin{equation} \label{eq:E_Cl}
E_{Cl_k}=\frac{\phi}{z_{Cl}}\ln\left( \frac{Cl_s}{Cl_k}\right) 
\end{equation}
%
Nernst potential for the NBC channel (mV):
\begin{equation} \label{eq:E_NBC}
E_{NBC_k}=\frac{\phi}{z_{NBC}}\ln\left( \frac{Na_s {HCO_{3_s}}^2}{Na_k {HCO_{3_k}}^2}\right) 
\end{equation}
Nernst potential for the \gls{bk} channel (mV):
\begin{equation} \label{eq:E_BK}
E_{BK_k}=\frac{\phi}{z_K}\ln\left( \frac{K_p}{K_k}\right) 
\end{equation}
%
Equilibrium state \gls{bk}-channel (-):
\begin{equation} \label{eq:winf}
w_{\infty}=0.5 \left(1+\mathrm{tanh}\left(\frac{v_{k}+v_{6} }{v_{4}} \right)  \right) 
\end{equation}
%
The time constant associated with the opening of the \gls{bk} channel (\psec):
\begin{equation}
\phi_n = \psi_n \cosh \left( \frac{v_k - v_3}{2 v_4} \right)
\end{equation}
%
The equilibrium state of the \gls{bk} channel (-):
\begin{equation}
w_{\infty} = \frac{1}{2} \left( 1 + \tanh \left( \frac{v_k + eet_{shift} eet_k - v_3}{v_4}  \right)  \right)
\end{equation}
%
The voltage associated with half open probability (mV):
\begin{equation}
v_3 = - \frac{v_5}{2} \tanh \left(  \frac{Ca_k - Ca_3}{Ca_4}  \right) + v_6
\end{equation}
%
The ratio $\rho$ of bound to unbound metabotropic receptors on the astrocytic process adjacent to the \gls{sc} (-):
\begin{equation}
\rho = \rho_{\text{min}} + \frac{\rho_{\text{max}} - \rho_{\text{min}}}{Glu_{max}}
Glu
\end{equation}
%
The ratio $G$ of active to total G-protein due to \gls{mglur} binding on the astrocyte endfoot surround the \gls{sc} (-):
\begin{equation}
G = \frac{\rho + \delta_G}{K_G + \rho + \delta_G}
\end{equation}
%
Fast \ca buffering is described within the steady state approximation (-):
\begin{equation}
B_{cyt} = \left( 1 + BK_{end} + \frac{K_{ex} B_{ex}}{(K_{ex} + Ca_k)^2}  \right)^{-1}
\end{equation}
%
The flux of \ca through the IP$_3$R channel (\mus):
\begin{equation}
J_{IP3_k} = J_{max} \left[  \left(\frac{IP3_k}{IP3_k + K_i}\right) \left(\frac{Ca_k}{Ca_k + K_{act_k}}\right) h_k \right]^3
\left( 1 - \frac{Ca_k}{s_k}  \right)
\end{equation}
%
The flux of \ca through the uptake pump (\mus):
\begin{equation}
J_{pump_k} = V_{max} \frac{Ca_k^2}{Ca_k^2 + k_{pump}^2}
\end{equation}
%
The flux of \ca through the leak channel (\mus):
\begin{equation}
J_{ERleak_k} = P_L \left(  1 - \frac{Ca_k}{s_k}  \right)
\end{equation}
%
\begin{longtable}[h!]{ p{0.1\linewidth}   p{0.68\linewidth}   p{0.22\linewidth} }
	\hline
	Parameter & Description & Value \\
	\hline
$VR_{sk}$ 			& Volume ratio between the \gls{sc} and astrocyte & 0.465 \\
$Glu_{max}$ 		& Maximum glutamate concentration (one vesicle) 		& 1846 $\mu$M   \\
$K_{e_{switch}}$	& Threshold past which glutamate vesicle is released		& 5.5 mM \\
$Glu_{slope}$		& Slope of glutamate sigmoidal				& 0.1 mM	 \\
$c_{unit}$			& Constant to convert from mM to \uM & $10^3$ \\
$k_{syn}$			& The number of active synapses per astrocytic process					& 11.5		 \\
$\gamma_j$				& Change in membrane potential by a scaling factor					& 1970 mV \uM\n \\
$G_{K_{k}}$ 	& Specific ion conductance of \pot 			& 6907.77 $\mu$M mV\n \psec  \\
$G_{Na_k}$ 		& Specific ion conductance of \na 		& 226.94  $\mu$M mV\n \psec   \\
$G_{NBC_k}$ 	& Specific ion conductance of the NBC cotransporter			& 130.74 $\mu$M mV\n \psec  \\
$G_{KCC1_k}$ 	& Specific ion conductance of the KCC1 cotransporter					& 1.728  $\mu$M mV\n \psec  \\
$G_{NKCC1_k}$ 	& Specific ion conductance of the NKCC1 cotransporter	 				& 9.568 $\mu$M mV\n \psec \\
$G_{BK_k}$ 		& Specific ion conductance of the \gls{bk} channel		& 10.25 $\mu$M mV\n \psec  \\
$G_{Cl_k}$ 		& Specific ion conductance of \cl 		&   151.93  $\mu$M mV\n \psec \\
$J_{NaK_{max}}$ & Maximum flux through the \sodpot \gls{atp}-ase pump							& 2.37$\e{4}$ \mus  \\
$K_{Na_k}$ & \sodpot \gls{atp}-ase pump constant							& $10\e{3}$ \uM  \\
$K_{K_s}$ & \sodpot \gls{atp}-ase pump constant							& $1.5\e{3}$ \uM  \\
$v_{6}$			& Voltage associated with the opening of half the population		& 22 mV  \\
$v_{4}$			& A measure of the spread of $w_{\infty}$	& 14.5 mV \\
$ \psi_{w}$    	& A characteristic time for the open probability of the \gls{bk} channel		& 2.664 \psec  \\
$\rho_{\text{min}}$ & Minimum ratio of bound to unbound \ip receptors  & 0.1  \\
$\rho_{\text{max}}$ & Maximum ratio of bound to unbound \ip receptors  & 0.7 \\
$\delta_G$ & Ratio of the activities of the unbound and bound receptors & $1.235 \e{-2}$ \\
$K_G$ & G-protein disassociation constant &  8.82 \\
$r_h$ & Maximum rate of \ip production in astrocyte due to glutamate receptors & 4.8 $\mu$M \psec  \\
$k_{deg}$ & Rate constant for \ip degradation in astrocyte  & 1.25 \psec  \\
$r_{buff}$ & Rate of \ca buffering at the endfoot compared to the astrocyte body  & 0.05  \\
$VR_{ERcyt}$ & Volume ratio between \gls{er} and astrocytic cytosol  & 0.185  \\
$BK_{end}$ & Ratio of endogenous buffer concentration to disassociation constant  & 40  \\
$K_{ex}$& Disassociation constant of exogenous buffer & 0.26 \uM  \\
$B_{ex}$& Concentration of exogenous buffer  & 11.35 \uM  \\
$J_{max}$ & Maximum rate of \ca through the \ip mediated channel  & 2880 \mus \\
$K_i$& Disassociation constant for \ip binding to an $IP_3R$  & 0.03 \uM  \\
$K_{act_k}$ & Disassociation constant for \ca binding to an activation site on an $IP_3R$  & 0.17 \uM \\
$k_{on}$ & Rate of \ca binding to the inhibitory site on the $IP_3R$  & 2 \uM\n \psec \\
$K_{inh}$ & Disassociation constant of $IP_3R$ & 0.1 \uM   \\
$V_{max}$ & Maximum rate of \ca uptake pump on the \gls{er}  & 20 \mus  \\
$k_{pump}$ & \ca uptake pump disassociation constant  & 0.24 \uM  \\
$P_L$ & \Gls{er} leak channel steady state balance constant  & 0.0804 \uM \psec \\			
$V_{eet}$& \Gls{eet} production rate & 72 \psec  \\
$c_{k_{min}}$& Minimum \ca concentration required for \gls{eet} production  & 0.1 \uM \\
$k_{eet}$ & \gls{eet} degradation rate  & 7.2 \psec  \\
$\psi_n$ & Characteristic time for the opening of the \gls{bk} channel & 2.664 \psec  \\
$v_4$ & Measure of the spread of $w_{\infty}$  & 8 mV \\
$eet_{shift}$& Describes the \gls{eet} dependent voltage shift  & 2 mV \uM\n   \\
$v_5$ & Determines the range of the shift of $w_{\infty}$ as \ca varies  & 15 mV \\
$v_6$& Shifts the range of $w_{\infty}$  & -55 mV  \\
$Ca_3$ & \Gls{bk} open probability \ca constant  & 0.4 \uM \\
$Ca_4$& \gls{bk} open probability \ca constant   & 0.35 \uM  \\			
			\hline
\caption{Parameters of the astrocyte and \gls{sc} submodel, for references see \citet{Dormanns2015, Kenny2018}.}
\end{longtable}

\section{\Gls{pvs}}
%
\subsection{\Glspl{ode}}
%
\pot concentration in the \gls{pvs}  (\uM):
\begin{equation} \label{eq:K_p}
\frac{dK_{p}}{dt}= \frac{J_{BK_k}}{VR_{pk}} + \frac{J_{KIR_i}}{VR_{pi}} - K_{decay_p} (K_p - K_{min_p})
\end{equation}
%
\ca concentration in the \gls{pvs} (\uM): 
\begin{equation}
\frac{d Ca_p}{dt} = \frac{J_{TRPV_k}}{VR_{pk}} + \frac{J_{VOCC_i}}{VR_{pi}} - Ca_{decay_p} ( Ca_p - Ca_{min_p} )
\end{equation}
%
The open probability of the \gls{trpv} channel (-): 
\begin{equation}
\frac{d m_k}{dt} = \frac{m_{\infty_k} - m_k}{t_{TRPV_k}} 
\end{equation}
%
		\subsection{Algebraic Variables}	
%
The flux of \ca through the \gls{vocc} which connects the \gls{smc} to the \gls{pvs} (\mus):
\begin{equation}
J_{VOCC_i} = G_{Cai} \frac{v_i - v_{Ca1}}{1 + \exp \left[-(v_i - v_{Ca2})/R_{Cai} \right]} 
\end{equation}
%
The flux of \ca through the \gls{trpv} channel (\mus):
\begin{equation}
J_{TRPV_k} = G_{TRPV_k} m_k (v_k - E_{TRPV_k})
\label{eq:Jtrpv}
\end{equation}
%			
The Nernst potential of the \gls{trpv} channel (mV):
\begin{equation}
E_{TRPV_k} = \frac{\phi}{z_{Ca}} \ln \left(\frac{Ca_p}{Ca_k} \right)
\end{equation}
%
The equilibrium state of the \gls{trpv} channel (-):
\begin{equation}
m_{\infty_k} =  \Gamma_m
				\left[
				\frac{1}{1 + H_{Ca_k}} \left( H_{Ca_k} + \tanh \left( \frac{v_k - v_{1,TRPV}}{v_{2,TRPV}} \right) \right) 
				\right] 
\end{equation}
%
The material strain gating term (-):
\begin{equation}
\Gamma_m = \frac{1}{1+\exp \left( {-\frac{\eta - \eta_0}{\kappa_k}} \right) }  
\end{equation}
%	
The strain on the perivascular endfoot of the astrocyte (-)
\begin{equation}
\eta = \frac{R - R_{init}}{R_{init}}
\end{equation}
%
The \ca inhibitory term (-)
\begin{equation}
H_{Ca_k} = \frac{Ca_k}{\gamma_{Cai}} + \frac{Ca_p}{\gamma_{Cae}}
\end{equation}
%
\begin{longtable}[h!]{ p{0.1\linewidth}   p{0.52\linewidth}   p{0.27\linewidth} }
	\hline
	Parameter & Description & Value \\
	\hline
$VR_{pk}$  & Volume ratio between \gls{pvs} and astrocyte   & 0.001 \\
$VR_{pi}$  & Volume ratio between \gls{pvs} and \gls{smc} & 0.001  \\
$K_{decay_p}$  & Rate of decay of \pot in \gls{pvs} & 0.15 \psec \\
$K_{min_p}$  & Steady state value of \pot in \gls{pvs} & $3\e{3}$ $\mu$M  \\
$Ca_{decay_p}$  & Rate of decay of \ca in \gls{pvs} & 0.5 \psec \\
$Ca_{min_p}$  & Steady state value of \ca in \gls{pvs} & $2\e{3}$ $\mu$M  \\
$G_{Cai}$  & \Gls{vocc} whole cell conductance & $1.29 \e{-3}$ $\mu$M mV\n \psec \\
$v_{Ca1}$  & \gls{vocc} reversal potential & 100 mV \\
$v_{Ca2}$ & Half point of the \gls{vocc} activation sigmoidal & $-24$ mV  \\
$R_{Cai}$ & Maximum slope of the \gls{vocc} activation sigmoidal & 8.5 mV \\
$G_{TRPV_k}$  & \Gls{trpv} whole cell conductance & $3.15 \e{-4}$ $\mu$M mV\n \psec \\
$t_{TRPV_k}$  & Characteristic time constant for $m_k$ & 0.9 s\\
$\eta_0$ & Strain required for half activation of the \gls{trpv} channel  & 0.1 \\
$\kappa_k$  & \gls{trpv} channel strain constant & 0.1  \\
$v_{1,TRPV}$ & \gls{trpv} channel voltage gating constant & 120 mV \\
$v_{2,TRPV}$ & \gls{trpv} channel voltage gating constant  & 13 mV \\
$\gamma_{Cai}$  & \ca concentration constant & 0.01 $\mu$M \\
$\gamma_{Cae}$& \ca concentration constant & 200 $\mu$M  \\
\hline
\caption{Parameters of the \gls{pvs} compartment, for references see \citet{Dormanns2015, Kenny2018}.}
		\end{longtable}

\section{\Gls{smc}}
%
\subsection{\Glspl{ode}}
%
Cytosolic \ca in the \gls{smc} (\uM):
\begin{equation}\label{eq:ci}
\begin{split}
\frac{dCa_i}{dt} = J_{IP_{3i}} - J_{SR_{uptake_{i}}} + J_{CICR_{i}} - J_{extrusion_{i}} +  J_{SR_{leak_{i}}}\dots \\
 - J_{VOCC_{i}} + J_{Na/Ca_{i}}  - 0.1J_{stretch_{i}} + J_{Ca^{2+}-coupling_{i}}^{SMC-EC}
\end{split} 
\end{equation}
%
\ca in the \gls{sr} of the \gls{smc} (\uM):
\begin{equation} \label{eq:si}
\frac{ds_i}{dt} =  J_{SR_{uptake_{i}}} - J_{CICR_{i}} - J_{SR_{leak_{i}}}
\end{equation}
%
Membrane potential of the \gls{smc} (mV):
\begin{equation} \label{eq:vi}
\begin{split}
\frac{dv_{i}}{dt} = \gamma_j( -J_{NaK_{i}} - J_{Cl_{i}} - 2J_{VOCC_{i}}- J_{Na/Ca_{i}} - J_{K_{i}}
- J_{stretch_{i}} - J_{KIR_{i}} ) +V^{SMC-EC}_{coupling_{i}}
\end{split}
\end{equation}
%
Open state probability of \ca-activated \pot channels (-):
\begin{equation} \label{eq:dwidt}
\frac{dw_{i}}{dt} =  \lambda_{i} \left( K_{act_{i}} - w_{i} \right)
\end{equation}
%
\ip concentration in the \gls{smc} (\uM):
\begin{equation} \label{eq:dIidt}
\frac{dIP3_i}{dt} = J^{SMC-EC}_{IP_{3}-coupling_{i}} - J_{degrad_{i}}
\end{equation}
%
\pot concentration in the \gls{smc} (\uM):
\begin{equation} \label{eq:dkidt}
\frac{d K_i}{dt}  = J_{NaK_{i}}  - J_{KIR_{i}} - J_{K_{i}}
\end{equation}
%
		\subsection{Algebraic Variables}	
%
Release of \ca from \ip sensitive stores in the \gls{smc} (\mus):
\begin{equation} \label{eq:IP3i}
J_{IP3_i} = F_{i}\frac{IP3_i^{2}}{K_{ri}^{2}+IP3_{i}^{2}}
\end{equation}
%
Uptake of \ca into the \gls{sr} (\mus):
\begin{equation} \label{eq:JSRuptakei}
J_{SR_{uptake_{i}}} = B_{i}\frac{Ca_i^{2}}{c_{bi}^{2}+Ca_i^{2}}
\end{equation}
%
\Gls{cicr} (\mus):
\begin{equation} \label{eq:JCICRi}
J_{CICR_{i}} = C_{i}\frac{\ s_i^{2}}{s_{ci}^{2}+\ s_i^{2}}    \frac{Ca_i^{4}}{c_{ci}^{4}+Ca_i^{4}}
\end{equation}
%
\ca extrusion by \ca-\gls{atp}-ase pumps (\mus):
\begin{equation} \label{eq:Jextrusioni}
J_{extrusion_{i}} = D_{i}Ca_i   \left( 1+ \frac{v_{i}-v_{d}}{R_{di}}\right)
\end{equation}
%
Leak current from the \gls{sr} (\mus):
\begin{equation} \label{eq:JSRleaki}
J_{SR_{leak_{i}}} = L_{i} s_i
\end{equation}
%
Flux of \ca exchanging with \na in the \na/\ca exchange (\mus): 
\begin{equation} \label{eq:JNaCai}
J_{Na/Ca_{i}} = G_{Na/Ca_{i}} \frac{Ca_i}     {Ca_i + c_{Na/Cai}} \left( v_{i}-v_{Na/Ca_{i}} \right)
\end{equation}
%
\ca flux through the stretch-activated channels in the \gls{smc} (\mus): 
\begin{equation} \label{eq:Jstretchi}
\begin{split}
J_{stretch_{i}}= \frac{G_{stretch}}{1+ \exp\left(-\alpha_{stretch}  \left(  \frac{\Delta pR}{h} -\sigma_{0}   \right) \right)}  \left(  v_{i}-E_{SAC}   \right) 
\end{split}
\end{equation}
%
Flux through the \sodpot pump (\mus): 
\begin{equation} \label{eq:J_NaK_i}
J_{NaK_{i}}= F_{NaK}
\end{equation}
%
\cl flux through the \cl channel (\mus):
\begin{equation} \label{eq:JCli}
J_{Cl_{i}} = G_{Cli} \left(  v_{i} - v_{Cli}  \right) 
\end{equation}
%
\pot flux through \pot channel (\mus):
\begin{equation} \label{eq:JKi}
J_{K_{i}}= G_{Ki} w_{i} \left(  v_{i} - v_{K_i}  \right) 
\end{equation}
%
Flux through \gls{kir} channels in the \gls{smc} (\mus): 
\begin{equation} \label{eq:JKIRi}
J_{KIR_{i}} =  G_{KIR_{i}}( v_{i} - v_{KIR_{i}})
\end{equation}
%
\ip degradation (\mus): 
\begin{equation} \label{eq:Jdegradi}
J_{degrad_{i}}= k_{di}I_{i}
\end{equation}
%
Nernst potential of the \gls{kir} channel in the \gls{smc} (mV):
\begin{equation}\label{eq:vKIR}
v_{KIR_i} = z_1 K_p-z_2
\end{equation}
%
Conductance of \gls{kir} channel (\uM mV\n \psec):
\begin{equation}\label{eq:gKIR}
G_{KIR_i} = F_{KIR_{i}} \exp(z_5 v_i +z_3 K_p - z_4)
\end{equation}
%
Equilibrium distribution of open channel states for the \gls{smc} \gls{bk} channels (-):
\begin{equation} \label{eq:Kacti}
K_{act_{i}} = \frac{  \left( Ca_i + c_{w,i}\right)^{2}}    {\left( Ca_i + c_{w,i} \right)^{2}    + \alpha_{act_i} \exp( -\left(   \left[ v_{i}-v_{Ca_{3i}}\right] /R_{Ki}   \right) )      }
\end{equation}
%
Translation factor, regulatory effect of \gls{cgmp} on the \gls{bk} channel open probability (\uM): 
\begin{equation} 
c_{\text{w},i} = \frac{\beta_{w,i}}{2} (1 + \tanh\left( \frac{\text{cGMP}_i - \alpha_{w,i}}{\epsilon_{w,i}}\right) )
\end{equation}
%
Heterocellular electrical coupling between \glspl{smc} and \glspl{ec} (mV \psec):
\begin{equation} \label{eq:Vcouplingi}
V_{coupling_{i}}^{SMC-EC}= -G_{coup}(v_{i}-v_{j})
\end{equation}
%
Heterocellular \ip coupling between \glspl{smc} and \glspl{ec} (\mus):
\begin{equation} \label{eq:JIP3couplingi}
J_{IP_{3}-coupling_{i}}^{SMC-EC}= -P_{IP_{3}}(IP3_{i}-IP3_{j})
\end{equation}
%
\ca coupling between \glspl{smc} and \glspl{ec} (\mus):
\begin{equation} \label{eq:JCAcouplingi}
J_{Ca^{2+}-coupling_{i}}^{SMC-EC}= -P_{Ca^{2+}}(Ca_i-Ca_j)
\end{equation}
%
\xspace 
%
\begin{longtable}[h!]{ p{0.12\linewidth}   p{0.58\linewidth}   p{0.26\linewidth} }
	\hline
	Parameter & Description & Value \\
	\hline
$\lambda_{i} $				& Rate constant for opening											& 45 \psec \\
 $F_{i}$      			& Maximal rate of activation-dependent \ca influx			& 0.23 \mus	\\
$K_{ri}$				& Half-saturation constant for agonist-dependent \ca entry	& 1 \uM	\\
$B_{i}$      			& \gls{sr} uptake rate constant							& 2.025 \mus \\
$c_{bi}$				& Half-point of the \gls{sr} \gls{atp}-ase activation sigmoidal 	& 1 \uM \\
$C_{i}$      			& \gls{cicr} rate constant									& 55 \mus \\
$s_{ci}$				& Half-point of the \gls{cicr} \ca efflux sigmoidal			& 2 \uM\\
$c_{ci}$				& Half-point of the \gls{cicr} activation sigmoidal			& 0.9 \uM \\
$D_{i}$      			& Rate constant for \ca extrusion by the \gls{atp}-ase pump		 & 0.24	\psec \\
$v_{d}$					& Intercept of voltage dependence of extrusion \gls{atp}-ase			 & $-100$ mV \\
$R_{di}$				& Slope of voltage dependence of extrusion \gls{atp}-ase				 & 250 mV	 \\
$L_{i}$      			& Leak from \gls{sr} rate constant						 & 0.025 \psec	 \\
$G_{Cai}$ 	& Whole-cell conductance for \glspl{vocc}	 	& 1.29$\times$10$^{-3}$  \uM mV\n \psec	\\
$v_{Ca_{1i}}$   & Reversal potential for \glspl{vocc}	 						& 100  mV \\
$v_{Ca_{2i}}$  	& Half-point of the \gls{vocc} activation sigmoidal		 	& $-24$ mV	 \\
$R_{Cai}$      	& Maximum slope of the \gls{vocc}	activation sigmoidal		& 8.5 mV	 \\
$G_{Na/Cai}$   	& Whole-cell conductance for \na/\ca exchange & 3.16$\times$10$^{-3}$ \uM mV\n \psec \\
$c_{Na/Cai}$   	& Half-point for activation of \na/\ca exchange by \ca		 & 0.5 \uM	 \\
$v_{Na/Cai}$   	& Reversal potential for the \na/\ca exchanger					 & -30 mV	 \\
$G_{stretch}$      		& Whole cell conductance for \glspl{sac}						& 6.1$\times$10$^{-3}$ \uM mV\n \psec \\
$\alpha_{stretch}$ & Slope of stress dependence of the \gls{sac} activation sigmoidal	& 7.4$\times$10$^{-3}$ mmHg$^{-1}$ \\
$ \Delta p $			& Pressure difference over vessel								& 30 mmHg	\\
$\sigma_{0}$      		& Half-point of the \gls{sac} activation sigmoidal				& 500 mmHg	 \\
$E_{SAC}$      			& Reversal potential for \glspl{sac}							& $-18$ mV \\
$F_{NaK}$      			& Rate of the \pot influx by the \sodpot pump & 4.32$\times$10$^{-2}$ \mus \\
$G_{Cli}$      			& Whole-cell conductance for \cl current		& 1.34$\times$10$^{-3}$ \uM mV\n \psec\\
$v_{Cli}$      			& Reversal potential for \cl channels		& $-25$  mV \\
$G_{Ki}$      	& Whole-cell conductance for \pot efflux		& 4.46$\times$10$^{-3}$ \uM mV\n \psec \\
$v_{K_i}$      	& Nernst potential										& $-94$ mV	 \\
$ F_{KIR_{i}} $ & Scaling factor of \pot efflux through the \gls{kir} channel & 0.381 \uM mV\n \psec  \\
$k_{di}$      			& Rate constant of \ip degradation	& 0.1 \psec \\
$\alpha_{act_i}$ & Translation factor for $v_i$ dependence of $K_{act_{i}}$ sigmoidal	& 0.13 \uM$^2$ \\
$v_{Ca_{3i}}$   		& Half-point for the K$_{Ca}$ channel activation sigmoidal		& $-27$ mV	\\
$R_{Ki}$      			& Maximum slope of the K$_{Ca}$ activation sigmoidal				& 12 mV \\
$ z_1 $	& Model estimation for membrane voltage \gls{kir} channel			  & 4.5$\times$10$^3$ mV \uM\n\\
$ z_2 $	& Model estimation for membrane voltage \gls{kir} channel			  & 112 mV \\
$ z_3 $	& Model estimation for the \gls{kir} channel conductance			  & $4.2\e{-4}$ \uM\n  \\
$ z_4 $	& Model estimation for the \gls{kir} channel conductance			  & 12.6 \\
$ z_5 $	& Model estimation for the \gls{kir} channel conductance	 & $-7.4\e{-2}$ mV\n  \\
$\beta_{w,i}$ & Constant to fit data & 1 \uM \\
$\alpha_{w,i}$ & Constant to fit data & 10.75 \uM \\
$\epsilon_{w,i}$ & Constant to fit data & 0.668 \uM \\
$G_{coup}$      		& Heterocellular electrical coupling coefficient		& 0.5 \psec	 \\
$P_{IP_{3}}$      		& Heterocellular \ip coupling coefficient	& 0.05 \psec\\
$P_{Ca^{2+}}$      		& Heterocellular $P_{Ca^{2+}}$ coupling coefficient	& 0.05 \psec \\
  \hline
\caption{Parameters of the \gls{smc} compartment, for references see \citet{Dormanns2015}.}
\end{longtable}

\section{\Gls{ec}}
%
\subsection{\Glspl{ode}}
%
Cytosolic \ca concentration in the \gls{ec} (\uM):
\begin{equation} \label{eq:cj}
\begin{split}
\frac{dCa_j}{dt} = J_{IP_{3j}} - J_{ER_{uptake_{j}}} + J_{CICR_{j}} - J_{extrusion_{j}}\dots \\
 + J_{ER_{leak_{j}}} + J_{cation_{j}} + J_{0_{j}} - J_{stretch_{j}} - J_{Ca^{2+}-coupling_{j}}^{SMC-EC}
\end{split}
\end{equation}
%
\ca concentration in the \gls{er} in the \gls{ec} (\uM): 
\begin{equation} \label{eq:sj}
\frac{ds_j}{dt} =  J_{ER_{uptake_{j}}} - J_{CICR_{j}} - J_{ER_{leak_{j}}}
\end{equation}
%
Membrane potential of the \gls{ec} (mV):
\begin{equation} \label{eq:dvjdt}
\frac{dv_{j}}{dt} =-\frac{1}{C_{m_{j}}} ( I_{K_{j}}+I_{R_{j}}) - V^{SMC-EC}_{coupling_{j}}
\end{equation}
%
\ip concentration of the \gls{ec} (\uM):
\begin{equation} \label{eq:dIjdt}
\frac{dIP3_{j}}{dt} =  J_{PLC}- J_{degrad_{j}}  - J^{SMC-EC}_{IP_{3}-coupling_{j}}
\end{equation}
%
		\subsection{Algebraic Variables}	
%
Release of \ca from \ip-sensitive stores in the \gls{ec} (\mus):
\begin{equation} \label{eq:JIP3j}
J_{IP_{3j}} = F_{j}\frac{IP3_{j}^{2}}{K_{rj}^{2}+IP3_{j}^{2}}
\end{equation}
%
Uptake of \ca into the endoplasmic reticulum (\mus):
\begin{equation} \label{eq:JERuptakej}
J_{ER_{uptake_{j}}} = B_{j}\frac{Ca_j^{2}}{c_{bj}^{2}+Ca_j^{2}}
\end{equation}
%
\Gls{cicr} (\mus):
\begin{equation} \label{eq:JCICRJ}
J_{CICR_{j}} = C_{j}\frac{s_j^{2}}{s_{cj}^{2}+s_j^{2}}    \frac{Ca_j^{4}}{c_{cj}^{4}+Ca_j^{4}}
\end{equation}
%
\ca extrusion by \ca-\gls{atp}-ase pumps (\mus):
\begin{equation} \label{eq:Jextrusionj}
J_{extrusion_{j}} = D_{j}Ca_j 
\end{equation}
%
\ca flux through the stretch-activated channels in the \gls{ec} (\mus): 
\begin{equation} \label{eq:Jstretchj}
J_{stretch_{j}}= \frac{G_{stretch}}{1+ \exp\left(-\alpha_{stretch}  \left(  \frac{\Delta pR}{h} -\sigma_{0}   \right) \right)}  \left(  v_{j}-E_{SAC}   \right) 
\end{equation}
%
Leak current from the \gls{er} (\mus):
\begin{equation} \label{eq:JERleakj}
J_{ER_{leak_{j}}} = L_{j}s_j
\end{equation}
%
\ca influx through nonselective cation channels (\mus):
\begin{equation} \label{eq:Jcationj}
J_{cation_{j}} = G_{cat_{j}} (E_{Ca_{j}} - v_{j}) \frac{1}{2} \left(   1+ \mathrm{tanh}  \left(  \frac{\mathrm{log}_{10} (Ca_j/c_{log})  - m_{3_{cat_{j}}} }    {m_{4_{cat_{j}}}}   \right)      \right) 
\end{equation}
%
\pot current through the $BK_{Caj}$ channel and the $SK_{Caj}$ channel (fA):
\begin{equation} \label{eq:JKj}
I_{K_{j}} = G_{totj} (v_{j}-v_{Kj}) \left(   I_{BK_{Caj}} + I_{SK_{Caj}} \right) 
\end{equation}
%
\pot efflux through the $BK_{Caj}$ channel (-):
\begin{equation} \label{eq:JBKCAj}
I_{BK_{Caj}} = 0.2 \left(   1+ \mathrm{tanh}   \left(   \frac{   (\mathrm{log}_{10} (Ca_j/c_{log}) - c) (v_{j}-b_{j}) - a_{1j}  }   { m_{3bj} ( v_{j} + a_{2j} (\mathrm{log}_{10} (Ca_j/c_{log}) -c )-b_{j} )^{2} + m_{4bj} }  \right)     \right)  
\end{equation}
%
\pot efflux through the $SK_{Caj}$ channel (-):
\begin{equation} \label{eq:JSKCaj}
I_{SK_{Caj}} = 0.3\left( 1+ \mathrm{tanh}  \left(  \frac{   \mathrm{log}_{10} (Ca_j/c_{log}) -m_{3sj}  } {m_{4sj}}  \right)      \right) 
\end{equation}
%
Residual current regrouping \cl and \na current flux (fA):
\begin{equation} \label{eq:JRj}
I_{R_{j}} = G_{R_{j}} ( v_{j} - v_{rest j}  )
\end{equation}
%
\ip degradation (\mus):  
\begin{equation} \label{eq:Jdegradj}
J_{degrad_{j}}= k_{dj} IP3_{j}
\end{equation}
%
\begin{longtable}[h!]{ p{0.12\linewidth}   p{0.6\linewidth}   p{0.28\linewidth} }
	\hline
	Parameter & Description & Value \\
	\hline
$J_{0_{j}}$ 			& Constant \ca influx & 0.029 \mus \\
$C_{m_{j}}$				& Membrane capacitance												& 25.8  pF \\
$ J_{PLC} $  & \ip production rate & 0.11 \mus  \\
$F_{j}$      			& Maximal rate of activation-dependent \ca influx			& 0.23 \mus			 \\
$K_{rj}$				& Half-saturation constant for agonist-dependent \ca entry	& 1 \uM	 \\
$B_{j}$      			& \gls{er} uptake rate constant							& 0.5 \mus	 \\
$c_{bj}$				& Half-point of the \gls{sr} \gls{atp}-ase activation sigmoidal 	& 1  \uM	 \\
$C_{j}$      			& \gls{cicr} rate constant									& 5 \mus \\
$s_{cj}$				& Half-point of the \gls{cicr} \ca efflux sigmoidal			& 2  \uM \\
$c_{cj}$				& Half-point of the \gls{cicr} activation sigmoidal			& 0.9 \uM \\
$D_{j}$      			& Rate constant for \ca extrusion by the \gls{atp}-ase pump		 & 0.24	\psec	\\
$L_{j}$      			& Rate constant for \ca leak from the \gls{er} 		 & 0.025	\psec \\
$G_{cat j}$      		& Whole-cell cation channel conductivity	& 6.6$\times$10$^{-4}$ \uM mV\n \psec	 \\
$E_{Caj}$      			& \ca equilibrium potential								 	& 50 mV	 \\
$c_{log}$				& Log constant & 1 \uM \\
$G_{totj}$      		& Total \pot channel conductivity	& 6927 pS \\
$v_{Kj}$      			& \pot equilibrium potential			& $-80$ mV	 \\
$m_{3_{catj}}$      	& Model constant, further explanation see \cite{Koenigsberger2006}		& $-0.18$	 \\
$m_{4_{catj}}$      	& Model constant, further explanation see \cite{Koenigsberger2006}		& 0.37	 \\
$c$      				& Model constant, further explanation see \cite{Koenigsberger2006}		& $-0.4$	 \\
$b_{j}$      			& Model constant, further explanation see \cite{Koenigsberger2006}		& $-80.8$ mV	 \\
$a_{1j}$      			& Model constant, further explanation see \cite{Koenigsberger2006}	& 53.3 mV \\
$a_{2j}$      			& Model constant, further explanation see \cite{Koenigsberger2006}		& 53.3 mV \\
$m_{3bj}$ & Model constant, further explanation see \cite{Koenigsberger2006} & 1.32$\times$10$^{-3}$ mV\n \\
$m_{4bj}$      			& Model constant, further explanation see \cite{Koenigsberger2006}		& 0.30 mV\\
$m_{3sj}$      			& Model constant, further explanation see \cite{Koenigsberger2006}	& $-0.28$	 \\
$m_{4sj}$      			& Model constant, further explanation see \cite{Koenigsberger2006}		& 0.389 \\
$G_{R_{j}}$      		& Residual current conductivity		& 955 pS		 \\
$v_{rest j}$      		& Membrane resting potential						 				& $-31.1$ mV	 \\
$k_{dj}$      			& Rate constant of \ip degradation						 		& 0.1 \psec \\
\hline
\caption{Parameters of the \gls{ec} compartment, for references see \citet{Dormanns2015}.}
\end{longtable}

\section{\Gls{no} pathway}
%
\subsection{\Glspl{ode}}
%
\ca concentration in the neuron (\uM): 
\begin{equation}       
\dfrac{dCa_n}{dt} = \frac{1}{1 + \lambda_{\text{buf}}}  \left( \frac{I_{\text{Ca,tot}}}{2 F V_{\text{spine}}} - \kappa_{\text{ex}}(Ca_n - [Ca]_{\text{rest}})
\right)
\end{equation}
%
Activated \gls{nnos} (\uM):
\begin{equation}  
\dfrac{d[\text{nNOS}]_n}{dt} = \frac{V_{\text{max,nNOS}} [\text{CaM}]_n}{K_{\text{m,nNOS}}+[\text{CaM}]_n}-\mu_{\text{deact},n} [\text{nNOS}]_n
\end{equation} 
%
\gls{no} concentration in the neuron (\uM):
\begin{equation}  
\dfrac{d\text{NO}_n}{dt} = p_{NO,n} - c_{NO,n} + d_{NO,n}
\end{equation}          	
%
\gls{no} concentration in the astrocyte (\uM):
\begin{equation}  
\dfrac{d\text{NO}_k}{dt} = p_{NO,k} - c_{NO,k} + d_{NO,k}
\end{equation}   
%
\gls{no} concentration in the \gls{smc} (\uM):
\begin{equation}  
\dfrac{d\text{NO}_i}{dt} = p_{NO,i} - c_{NO,i} + d_{NO,i} 
\end{equation}
%
Activated \gls{enos} (\uM):
\begin{equation} 
\dfrac{d[\text{eNOS}]_j}{dt} = \gamma_{\text{eNOS}} \frac{K_{\text{dis}}Ca_j}{K_{\text{m,eNOS}}+Ca_j} + (1-\gamma_{\text{eNOS}}) g_{\max} F_{\text{wss}}   - \mu_{\text{deact},j}[\text{eNOS}]_j
\end{equation}	
%
\gls{no} concentration in the \gls{ec} (\uM):
\begin{equation} 
\dfrac{d\text{NO}_j}{dt} = p_{NO,j} - c_{NO,j} + d_{NO,j} 
\end{equation}
%
Fraction of \gls{sgc} in the basal state (-):% Yang2005
\begin{equation} 
\dfrac{dE_b}{dt} = -k_1 E_b \text{NO}_i + k_{-1} E_{6c} + k_4 E_{5c}
\end{equation}	
%
Fraction of \gls{sgc} in the intermediate form (-):% Yang2005
\begin{equation} 
\dfrac{dE_{6c}}{dt} = k_1 E_b \text{NO}_i - (k_{-1} + k_2) E_{6c} - k_3 E_{6c} \text{NO}_i
\end{equation}	
%
Concentration of \gls{cgmp} in the \gls{smc} (\uM):		% Yang2005	
\begin{equation} 
\dfrac{d\text{cGMP}_i}{dt} = V_{\text{max,sGC}} E_{5c} - V_{\text{max,pde}} \frac{\text{cGMP}_i}
{K_{\text{m,pde}} + \text{cGMP}_i}
\end{equation}	
%
		\subsection{Algebraic Variables}
%
Fraction of open NR2A \gls{nmda} receptors (-): 
\begin{equation} 
w_{\text{NR2},A} = \frac{Glu}{K_{m,A} + Glu}
\end{equation}	
%
Fraction of open NR2B \gls{nmda} receptors (-):
\begin{equation} 
w_{\text{NR2},B} = \frac{Glu}{K_{\text{m},B} + Glu}
\end{equation}	
%
Inward \ca current per open \gls{nmda} receptor (fA):	
\begin{equation} 
I_{\text{Ca}} = \frac{4 v_n G_\text{M} (P_{\text{Ca}}/P_\text{M})([Ca]_{\text{ex}}/[\text{M}])}{1+ \exp(\alpha_v(v_n+\beta_v))} \frac{\exp(2v_n/\phi))}{1-\exp(2v_n/\phi))}
\end{equation}
%
Total inward \ca current for all open \gls{nmda} receptors per synapse (fA): 
\begin{equation} 
I_{\text{Ca,tot}} = (n_{\text{NR2},A} w_{\text{NR2},A} + n_{\text{NR2},B}  w_{\text{NR2},B}) I_{\text{Ca}} 
\end{equation}	
%
\ca-calmodulin complex concentration (\uM):
\begin{equation} 
[\text{CaM}]_n = \frac{Ca_n}{m_c}
\end{equation}
%
Neuronal \gls{no} production flux (\mus): 	
\begin{equation} 
p_{NO,n} = V_{\text{max,NO},n} [\text{nNOS}]_n \frac{[O_2]_n}{K_{\text{m,O2},n}+[O_2]_n} \frac{[LArg]_n}{K_{\text{m,LArg},n}+[LArg]_n}
\end{equation}	
%
Neuronal \gls{no} consumption flux (\mus): 				
\begin{equation} 
c_{NO,n} = k_{\text{O2},n} [\text{NO}]_n^2 [O_2]_n
\end{equation}
%
Neuronal \gls{no} diffusive flux (\mus): 	
\begin{equation} 
d_{NO,n} = \frac{[\text{NO}]_k - [\text{NO}]_n}{\tau_{nk}}
\end{equation}	
%
Time for \gls{no} to diffuse between the centres of the neuron and the astrocyte (s):
\begin{equation}
\tau_{nk} = \frac{x_{nk}^2}{2 D_{\text{c,NO}}}
\end{equation}
%
Astrocytic \gls{no} production flux (\mus):
\begin{equation} 
p_{NO,k} = 0
\end{equation}
%
Astrocytic \gls{no} consumption flux (\mus):
\begin{equation} 
c_{NO,k} = k_{\text{O2},k} [\text{NO}]_k^2 [O_2]_k
\end{equation}
%
Astrocytic \gls{no} diffusive flux (\mus):
\begin{equation} 
d_{NO,k} = \frac{[\text{NO}]_n - [\text{NO}]_k}{\tau_{nk}} + \frac{[\text{NO}]_i - [\text{NO}]_k}{\tau_{ki}}
\end{equation}	
%
Time for \gls{no} to diffuse between the centres of the astrocyte and the \gls{smc} (s):
\begin{equation}
\tau_{ki} = \frac{x_{ki}^2}{2 D_{\text{c,NO}}}
\end{equation}
%
\gls{smc} \gls{no} production flux (\mus):
\begin{equation} 
p_{NO,i} = 0
\end{equation}
%
\gls{smc} \gls{no} consumption flux (\mus):
\begin{equation} 
c_{NO,i} = k_{\text{dno}} [\text{NO}]_i
\end{equation}
%
\gls{smc} \gls{no} diffusive flux (\mus):
\begin{equation} 
d_{NO,i} = \frac{[\text{NO}]_k - [\text{NO}]_i}{\tau_{ki}} + \frac{[\text{NO}]_j - [\text{NO}]_i}{\tau_{ij}}
\end{equation}
%
\gls{sgc} kinetics rate constant (s\n): % Yang2005
\begin{equation} 
k_4 = C_4 [\text{cGMP}]_i^{2}
\end{equation}	
%
Fraction of \gls{sgc} in the fully activated form (-):% Yang2005
\begin{equation} 
E_{5c} = 1 - E_b - E_{6c}
\end{equation}	
%
Regulatory effect of \gls{cgmp} on myosin dephosphorylation (-):			%not on the BK channel open probability !
\begin{equation} 
R_{\text{cGMP}} = \frac{[\text{cGMP}]_i^2}{K_{\text{m,mlcp}}^2 + [\text{cGMP}]_i^2}
\end{equation}
%	
Maximum \gls{cgmp} production rate (\mus):
\begin{equation}
V_{\text{max,pde}} = k_{\text{pde}} [\text{cGMP}]_i
\end{equation}
%
Time for \gls{no} to diffuse between the centres of the \gls{smc} and the \gls{ec} (s):
\begin{equation}
\tau_{ij} = \frac{x_{ij}^2}{2 D_{\text{c,NO}}}
\end{equation}
%
Fraction of the elastic strain energy stored within the membrane (-): 
\begin{equation} 
F_{\text{wss}} = \frac{1}{1+\alpha_{\text{wss}} \exp(-W_{\text{wss}})} - \frac{1}{1+\alpha_{\text{wss}}}
\end{equation}	
%
Strain energy density (-): 
\begin{equation} 
W_{\text{wss}} = W_0 \frac{(\tau_{\text{wss}} + \sqrt{16 \delta_{\text{wss}}^2 + \tau_{\text{wss}}^2} - 4 \delta_{\text{wss}})^2}{\tau_{\text{wss}} + \sqrt{16\delta_{\text{wss}}^2 + \tau_{\text{wss}}^2}}
\end{equation}	
%
Wall shear stress (Pa): % unit??
\begin{equation}
\tau_{\text{wss}} = \frac{R \Delta P}{2 L}
\end{equation}
%
\ox concentration in the \gls{ec} (\uM):
\begin{equation}
[O_2]_j =  c_{unit} \, O_2 
\end{equation}
\gls{ec} \gls{no} production flux (\mus):
\begin{equation} 
p_{NO,j} = V_{\text{max,NO},j} [\text{eNOS}]_j  \frac{[O_2]_j}{K_{\text{m,O2},j}+[O_2]_j} \frac{[LArg]_j}{K_{\text{m,L-Arg},j}+[LArg]_j}
\end{equation}	
%
\gls{ec} \gls{no} consumption flux (\mus):
\begin{equation} 
c_{NO,j} = k_{\text{O2},j} [\text{NO}]_j^2 [O_2]_j 
\end{equation}	
%
\gls{ec} \gls{no} diffusive flux (\mus):					
\begin{equation} 
d_{NO,j} = \frac{[\text{NO}]_i - [\text{NO}]_j}{\tau_{ij}} - \frac{4 D_{\text{c,NO}}[\text{NO}]_j}{r_l^2}
\end{equation}		
%
\begin{longtable}[h!]{ p{0.12\linewidth}   p{0.64\linewidth}   p{0.24\linewidth} }
	\hline
	Parameter & Description & Value \\
	\hline
$ [\text{Glu}]_{\max} $			& Maximum glutamate concentration 						& 1846 \uM 	\\ 
$ V_{\text{spine}} $		& Dendritic spine volume 								& $8\e{-5}$ pL		\\ 
$ \kappa_{\text{ex}} $		& Decay rate constant of internal \ca concentration	& $1.6\e{3}$ s\n \\
$ [Ca]_{\text{rest}} $		& Resting internal \ca concentration				& 0.1 \uM		\\
$ \lambda_{\text{buf}} $	& Buffer capacity										& 20 	\\
$ V_{\text{max,nNOS}} $		& Maximum \gls{nnos} activation rate							& $25\e{-3}$ \uM \psec	\\
$ K_{\text{m,nNOS}} $		& Michaelis constant									& $9.27\e{-2}$ \uM	\\			
$ \mu_{\text{deact},n} $	& Rate constant at which \gls{nnos} is deactivated 			& 0.0167 s\n	\\ 
$ K_{\text{m},A} $			& Michaelis constant & 650 \uM \\ 
$ K_{\text{m},B} $			& Michaelis constant  & 2800 \uM \\
$ v_n $						& Neuronal membrane potential							& $-40$ mV	 \\
$ G_\text{M} $				& Conductance of \gls{nmda} receptor  						& $46$ pS	\\ 
$ P_{\text{Ca}}/P_\text{M} $ & Ratio of \ca permeability to monovalent ion permeability & 3.6 \\
$ [Ca]_{\text{ex}} $   	& External \ca concentration						& $2\e{3}$ \uM  \\
$ [\text{M}] $ 				& Concentration of monovalent ions 						& $1.3\e{5}$ \uM  \\
$ \alpha_v $				& Voltage-dependent Mg$^{2+}$ block parameter			& $-0.08$ mV\n	\\ 
$ \beta_v $					& Voltage-dependent Mg$^{2+}$ block parameter			& 20 mV \\ 
$ n_{\text{NR2},A} $		& Average number of NR2A \gls{nmda} receptors 	& 0.63  \\
$ n_{\text{NR2},B} $		& Average number of NR2B \gls{nmda} receptors 	& 11   \\
$m_c$						& Number of \ca ions bound per calmodulin & 4 \\
$ V_{\text{max,NO},n} $ 	& Maximum catalytic rate of neuronal \gls{no} production 		& 4.22 s\n 	\\
$ [O_2]_n $ 				& \ox concentration in the neuron 					& 200 \uM 	\\
$ K_{\text{m,O2},n} $ 		& Michaelis constant for \gls{nnos} for \ox  				& 243 \uM \\
$ [LArg]_n $ 				& L-Arg concentration in the neuron 					& 100 \uM 	\\ 
$ K_{\text{m,LArg},n} $ 	& Michaelis constant for \gls{nnos} for LArg 				& 1.5 \uM 	 \\
$ k_{\text{O2},n} $ 		& \ox reaction rate constant							& $9.6\e{-6}$ \uM$^{-2}$ s\n \\
$ x_{nk} $ 					& Distance between centres of neuron and astrocyte  	& 25 \um  \\ 
$ k_{\text{O2},k} $ 	& \ox reaction rate constant							& $9.6\e{-6}$\uM$^{-2}$s\n \\ 
$[O_2]_k$ 				& \ox concentration in the astrocyte & 200 \uM \\
$ x_{ki} $ 				& Distance between centres of astrocyte and \gls{smc} compartments 	& 25 \um \\ 
$ k_{-1} $ 				& \gls{sgc} kinetics rate constant 	& 100 s\n 			 \\ 
$ k_1 $ 				& \gls{sgc} kinetics rate constant 	& $2\e{3}$ \uM\n\ s\n 	 \\ 
$ k_2 $ 				& \gls{sgc} kinetics rate constant 	& 0.1 s\n 			 \\ 
$ k_3 $ 				& \gls{sgc} kinetics rate constant 	& 3 \uM\n\ s\n 		 \\ 
$ V_{\text{max,sGC}} $ 	& Maximal \gls{cgmp} production rate	& 0.8520 \mus  	 \\ 
$ K_{\text{m,pde}} $ 	& Michaelis constant 			& 2 \uM 			 \\ 
$ k_{\text{dno}}$ 		& Constant reflecting the activity of various \gls{no} scavengers & 0.01 s\n \\ 
$ C_4 $ 				& \gls{sgc} rate scaling constant 						& 0.011 \uM$^{-2}$ \psec \\ 
$ K_{\text{m,mlcp}} $ 	& Hill coefficient				& 5.5 \uM 			 \\
$ v_{\text{Ca}3,i} $ 			& Half-point for the $ K_{\text{Ca}} $ channel activation sigmoidal & $-27$ mV \\ 
$ R_{\text{K},i}  $ 				& Maximum slope of the $K_{Ca}$ activation sigmoidal & 12 mV \\ 
$ k_{\text{pde}} $		& Phosphodiesterase rate constant & 0.0195 s\n \\
$ x_{ij} $ 				& Distance between centres of \gls{smc} and \gls{ec} compartments 	& 3.75 \um \\
$ \gamma_{\text{eNOS}} $	& Relative strength of the \ca dependent pathway for \gls{enos} activation	& 0.1 	\\
$ \mu_{\text{deact},j} $	& \gls{enos}-caveolin association rate											& 0.0167 s\n	\\
$ K_{\text{dis}} $			& \gls{enos}-caveolin disassociation rate											& 0.09 \mus	\\
$ K_{\text{m,eNOS}} $		& Michaelis constant														& 0.45 \uM		\\
$ g_{\max} $		& Maximum wall-shear-stress-induced \gls{enos} activation			& 0.06 \mus	\\
$ \alpha_{\text{wss}} $				& Zero shear open channel constant						& 2 		\\
$ W_0 $					& Shear gating constant 									& 1.4 Pa\n			\\
$ \delta_{\text{wss}} $		& Membrane shear modulus			& 2.86 Pa		\\
$ V_{\text{max,NO},j} $ 	& Maximum catalytic rate of \gls{no} production		& 1.22 s\n		\\ 
$ K_{\text{m,O2},j} $		& Michaelis constant for \gls{enos} for \ox  		& 7.7 \uM 	 \\
$ [LArg]_j $				& L-Arg concentration in the neuron 		& 100 \uM 	 \\ 
$ K_{\text{m,L-Arg},j} $	& Michaelis constant for L-Arg 			& 1.5 \uM  \\
$ \Delta P / L $			& Pressure drop over length of arteriole	& $9.1\e{-2}$ Pa \um\n 	\\
$ k_{\text{O2},j} $ 		& \ox reaction rate constant			& $9.6\e{-6}$ \uM$^{-2}$ s\n \\ 
$ D_{\text{c,NO}} $			& \Gls{no} diffusion coefficient	& 3300 \um$^2$s\n  \\
$r_l$	& Constant of lumen radius & 25 \um \\
\hline
\caption{Parameters of the \gls{no} submodel, for references see \citet{Dormanns2016}.}
\end{longtable}	


\section{Wall mechanics}

\subsection{\Glspl{ode}}
Fraction of free phosphorylated cross-bridges (-):
\begin{equation} \label{eq:dMpdt}
\frac{d[Mp]}{d t} =  \chi_w 
\left( K_{4}[AMp] +K_{1} [M] - ( K_{2} + K_{3} ) [Mp]
\right) 
\end{equation}
%
Fraction of attached phosphorylated cross-bridges (-):
\begin{equation} \label{eq:dAMpdt}
\frac{d[AMp]}{d t} = \chi_w 
\left( K_{3} [Mp] + K_{6} [AM] - ( K_{4} + K_{5} )[AMp] \right)
\end{equation} 
%
Fraction of attached dephosphorylated cross-bridges (-):
\begin{equation} \label{eq:dAMdt}
\frac{d[AM]}{d t} =  \chi_w 
\left(  K_{5} [AMp]-(K_{7}+K_{6})[AM] \right)
\end{equation}
%
Vessel radius (\um):
\begin{equation} \label{eq:dRdt2e}
\frac{dR}{dt}= \frac{R_{init}}{\eta}\left(   \frac{ R P_{T}}{h}  - E \frac{R - R_0}{R_0} \right)
\end{equation}
%
		\subsection{Algebraic Variables}
Fraction of free non-phosphorylated cross-bridges (-):
\begin{equation} \label{eq:dMdt}
[M]=1-[AM]-[AMp]-[Mp]
\end{equation}
%
Rate constants for phosphorylation of M to Mp and of AM to AMp (\psec):
\begin{equation} \label{eq:gamma}
K_{1} = K_{6} = \gamma_{cross} Ca_i ^{n_{cross}}
\end{equation}
%
Rate constants for dephosphorylation of Mp to M and of AMp to AM (\psec):
\begin{equation} 
K_{2} = K_{5} = \delta_K \left(k_{\text{mlpc,b}} + k_{\text{mlpc,c}} R_{\text{cGMP}}\right)
\end{equation}	
%
Wall thickness of the vessel (in \um):
\begin{equation} \label{eq:h2}
h=0.1 R
\end{equation}
%
Fraction of attached myosin cross-bridges (-):
\begin{equation}
F_r = [AM_p] + [AM]
\end{equation}
%
Young's modulus (Pa):
\begin{equation}
E = E_{pas} + F_r \left(E_{act} - E_{pas} \right)
\end{equation}
%
Initial radius (\um):
\begin{equation}
R_0 = R_{init} + F_r (\alpha -1) R_{init}
\end{equation}
%
\begin{longtable}[h!]{ p{0.1\linewidth}   p{0.64\linewidth}   p{0.26\linewidth} }
	\hline
	Parameter & Description & Value \\
	\hline
$\chi_w$ 		& Scaling constant for wall mechanics & 1.7 \\
$K_{3}$      	& Rate constant for attachment of phosphorylated crossbridges	 & 0.4 \psec \\
$K_{4}$      	& Rate constant for detachment of phosphorylated crossbridges 	 & 0.1 \psec \\
$K_{7}$      	& Rate constant for detachment of dephosphorylated crossbridges 	& 0.1 \psec	 \\
$\gamma_{cross}$    & Sensitivity of the contractile apparatus to \ca		& 17 \uM$^{-3}$\psec \\
$n_{cross}$      	& Fraction constant of the phosphorylation crossbridge				& 3  \\
$ \delta_K $ 			& Constant to fit data			& 58.14 \\
$ k_{\text{mlpc,b}} $ & Basal MLC dephosphorylation rate constant			& $8.6\e{-3}$ \psec  \\
$ k_{\text{mlpc,c}} $ & First-order rate constant for \gls{cgmp} regulated MLC dephosphorylation & $32.7\e{-3}$ \psec  \\
$\eta$			& Viscosity															& 10$^4$ Pa s \\
$P_T$				& Transmural pressure												& 4$\times$10$^3$ Pa	 \\
${E}_{pas}$			& Young's moduli for the passive vessel								& 66$\times$10$^3$ Pa 	\\
${E}_{act}$			& Young's moduli for the active vessel								& 233$\times$10$^3$ Pa \\
$\alpha$			& Scaling factor for initial radius										& 0.6   \\
\hline
\caption{Parameters of the wall mechanics submodel, for references see \citet{Dormanns2015}.}
\end{longtable}

\section{Tissue Slice Model}

\subsection{\Glspl{ode}}
%
\pot concentration in the astrocyte of block $i$ with four neighbours $j$ (\uM):
\begin{align}
\frac{d K_k^i}{dt} = \sum_j Q_{K,j \to i} - J_{K_k}^i + 2 J_{NaK_{k}}^i + J_{NKCC1_{k}}^i +  J_{KCC1_{k}}^i
\end{align}
%
Membrane potential in the astrocyte of block $i$ with four neighbours $j$ (mV):
\begin{align}
\frac{d v_k}{dt} = \gamma_j \left[ \sum_j z_K Q_{K,j \to i} -J_{BK_k}^i - J_{K_k}^i - J_{Cl_k}^i - J_{NBC_k}^i - J_{Na_k}^i - J_{NaK_k}^i - 2J_{TRPV_k}^i \right]
\end{align}
%
\pot concentration in the \gls{ecs} of block $i$ with four neighbours $j$ (mM):
\begin{align}
\frac{dK_{e}^i}{dt} &= \sum_j Q^{e}_{K,j\to i} + \frac{1}{F f_e} \left( \frac{A_s I_{K,tot_{sa}}^i}{V_s} + \frac{A_d I_{K,tot_d}^i}{V_d} \right) - \frac{d \text{Buff}^i_e}{dt}
\end{align}
%
\na concentration in the \gls{ecs} of block $i$ with four neighbours $j$ (mM):
\begin{align}
\frac{d Na_e}{dt} &= \sum_j Q^{e}_{Na,j\to i} + \frac{1}{F f_e} \left( \frac{A_s I_{Na,tot_{sa}}^i}{V_s} + \frac{A_d I_{Na,tot_d}^i}{V_d} \right)
\end{align}
%
		\subsection{Algebraic Variables}
Rate of change of $K_k$ in block $i$ due to gap junctional flux from neighbour $j$ (\mus):
\begin{align}
Q_{K,j \to i} = \frac{D_{gap}}{\Delta x^2} \left( (K_k^j - K_K^i) + \frac{z_K F}{RT} \frac{K_k^i + K_k^j}{2} (v_k^j - v_k^i) \right)
\end{align}
%
Rate of change of $K_e$ and $Na_e$ in block $i$ due to extracellular electrodiffusion from neighbour $j$ (mM \psec):
\begin{align}
Q^{e}_{K,j\to i} &= \frac{D_{K,e}}{\Delta x^2} \left[ 
(K_e^j - K_e^i) - z_K \left(\frac{K_e^i + K_e^j}{2}\right) 
\left(   \ddfrac{z_K D_{K,e} (K_e^j - K_e^i) + z_{Na} D_{Na,e} (Na_e^j - Na_e^i)}
{z_K^2 D_{K,e} \frac{K_e^i + K_e^j}{2} + z_{Na}^2 D_{Na,e} \frac{Na_e^i + Na_e^j}{2} }
\right) 
\right]   
\\
Q^{e}_{Na,j\to i} &= \frac{D_{Na,e}}{\Delta x^2} \left[ 
(Na_e^j - Na_e^i) - z_{Na} \left(\frac{Na_e^i + Na_e^j}{2}\right) 
\left(   \ddfrac{z_K D_{K,e} (K_e^j - K_e^i) + z_{Na} D_{Na,e} (Na_e^j - Na_e^i)}
{z_K^2 D_{K,e} \frac{K_e^i + K_e^j}{2} + z_{Na}^2 D_{Na,e} \frac{Na_e^i + Na_e^j}{2} }
\right) 
\right]   
\end{align}
\pagebreak
\begin{longtable}[h!]{ p{0.12\linewidth}   p{0.66\linewidth}   p{0.22\linewidth} }
	\hline
	Parameter & Description & Value \\
	\hline
$D_{gap}$ 		& Astrocytic gap junction diffusion coefficient & $3.1\e{-9}$ m$^2$\psec \\
$\Delta x^2$ 	& Width of one \gls{nvu} block & $1.24\e{-4}$ m \\
$D_{K,e}$ 		& Extracellular \pot diffusion coefficient & $3.8\e{-9}$ m$^2$\psec \\
$D_{Na,e}$ 		& Extracellular \na diffusion coefficient & $2.5\e{-9}$ m$^2$\psec \\
\hline
\caption{Parameters of the large scale tissue slice model.}
\end{longtable}

\section*{References}
\bibliographystyle{apalike}
\addcontentsline{toc}{chapter}{Bibliography}
\bibliography{library}
\end{document}